\documentclass[12pt]{article}

\input{../preamble_for_assignments}
\input{4mba1q} % questions

%%%%%%%%%%%%%%%%%%%%%%%%%%%%%%%%%%%
%% FANCY HEADER AND FOOTER STUFF %%
%%%%%%%%%%%%%%%%%%%%%%%%%%%%%%%%%%%
\usepackage{fancyhdr,lastpage}
\pagestyle{fancy}
\fancyhf{} % clear all header and footer parameters
%%%\lhead{Student Name: \theblank{4cm}}
%%%\chead{}
%%%\rhead{Student Number: \theblank{3cm}}
%%%\lfoot{\small\bfseries\ifnum\thepage<\pageref{LastPage}{CONTINUED\\on next page}\else{LAST PAGE}\fi}
\lfoot{}
\cfoot{{\small\bfseries Page \thepage\ of \pageref{LastPage}}}
\rfoot{}
\renewcommand\headrulewidth{0pt} % Removes funny header line
%%%%%%%%%%%%%%%%%%%%%%%%%%%%%%%%%%%

\begin{document}
%Title
\begin{center}
{\bfseries Mathematics 4MB3/6MB3 Mathematical Biology\\
\smallskip
2019 ASSIGNMENT {\color{blue}1}}\\
\medskip
\underline{\emph{Group Name}}: \texttt{{\color{blue}The Plague Doctore}}\\
\medskip
\underline{\emph{Group Members}}: {\color{blue}Sid Reed, Daniel Segura, Jessa Mallare, Aref Jadda}
\end{center}
\section{Analysis of the SI model}
\SIanalIntro
\begin{enumerate}[(a)]
\item
\item
  \begin{enumerate}[(i)]
  \item
  \item
  \end{enumerate}
\end{enumerate}
\section{Analysis of the basic SIR  model}
\begin{enumerate}[(a)]
\item %2 a)
The peak prevalence of the disease will be the maximum number of people infected at a given time $t$ during the infection.
%proof
This implies that the peak prevalence is the maximum value of the function $I(t)$.
In order to get a function of $I$ with the initial conditions $(S_0,I_0)$ as parameters we can consider the phase portrait solution:
$$ I_{max} = I_0 + S_0 - \frac{1}{\mathcal R_0} + \frac{1}{\mathcal R_0}log(\frac{1}{S_0\mathcal R_0}))$$
which can be derived as follows:
%I(S) derivation
\begin{align*}
    \frac{dI}{dS}     &= \frac{dI/dt}{dS/dt}\\
                      &= -1 + \frac{1}{S\mathcal R_0}\\
    \int_{I_0}^{I} dI &= \int_{S_0}^{S} -1 + \frac{1}{S\mathcal R_0}dS\\
    I - I_0           &= -(S - S_0) + \frac{1}{\mathcal R_0}log(\frac{S}{S_0})\\
    I                 &= I_0 + S_0 - S + \frac{1}{\mathcal R_0}log(\frac{S}{S_0})
\end{align*}
%max val of I(S)
All maxima and minima of a function occur at points $(x,y)$ such that $f'(x) = 0$, thus
\begin{align*}
    \frac{dI/dt}{dS/dt} &= \frac{{\mathcal R_0}SI - I}{-\mathcal R_0SI}\\
    \frac{dI/dt}{dS/dt} &= -1 + \frac{1}{S\mathcal R_0}\\
    0                   &= -1 + \frac{1}{S\mathcal R_0}\\
    1                   &= \frac{1}{S\mathcal R_0}\\
    1                   &= S{\mathcal R_0}
\end{align*}
This equation is true when $S = \frac{1}{\mathcal R_0}$, thus the maximum value for the function $I(S)$ occurs when $S = \frac{1}{\mathcal R_0}$.
Substituting this into $I(S)$ will give an expression for $I_{max}$ in terms of the initial conditions $(S_0,I_0)$
$$ I_{max} = I_0 + S_0 - \frac{1}{\mathcal R_0} + \frac{1}{\mathcal R_0}log(\frac{1}{S_0\mathcal R_0}))$$
%Justification
This quantity may be important to a public health officials for triage. If an epidemic is expected to have a low peak prevalence, fewer health-related resources would need to be allocated to treatment and prevention.
If the peak prevalence is estimated to be high then more effort may be directed towards prevention and treatment of the disease.
Further if the peak prevalence is estimated the time of peak prevalence can be derived easily using the $I(t)$ function, to estimate how much time exists to prepare for the peak of the infection, when resources (money, health personnel, equipment, etc.) will be most strained.
\item %2 b)
  \begin{enumerate}[(i)]
      \item % i)
      \item % ii)
      \item % iii)
      \item % iv)
  \end{enumerate}
\item %2 c)
\item %2 d)
All points $(S,0) S \in 0 \leq S \leq 1$ are equilibria for the SIR model.
The jacobian of the system is
$$ DF_{(S,I)} =
\begin{bmatrix}
    -{\mathcal R_0}I & -{\mathcal R_0}S\\
     {\mathcal R_0}I &  {\mathcal R_0}S - 1\\
\end{bmatrix} $$
Substituting in the equilibrium point gives:
$$ DF_{(S,0)} =
\begin{bmatrix}
    0 & -{\mathcal R_0}S\\
    0 &  {\mathcal R_0}S - 1\\
\end{bmatrix} $$
The eigenvalues of the matrix are the roots of the equation $\lambda^2 - T\lambda + D$ where $T$ and $D$ are the trace and determinant respectively.
\begin{align*}
    T &= {\mathcal R_0}S - 1\\
    D &= 0\\
    0 &= \lambda^2 - ({\mathcal R_0}S - 1)\lambda + (0)\\
    0 &= \lambda({\mathcal R_0}S - 1 + \lambda)\\\\
    \lambda &= 0\qquad or\qquad \lambda &= -({\mathcal R_0}S - 1)
\end{align*}
For one of the eigenvalues $R(\lambda) = 0$, therefore the equilibria are non-hyperbolic and the stability must be assessed in some other way.
%Lyapunov explanation
This assessment can be done by examining a Lyapunov function.
Lyapunov's theorem states that foran equilibrium point $X_*$ of $X'=F(X)$ and some set ${\mathcal S}$ if $\exists L(X)$ such that
\begin{align*}
    L(X_*) &= 0\\
    L(X) &\geq 0\quad \forall X \in S\char`\\ \{X_*\}\\
    \nabla L(X) \cdot X' &< 0 \quad \forall X \in S\char`\\ \{X_*\}
\end{align*}
then $L(X)$ is a strict Lyapunov function and $X_*$ is asymptotically stable.
%Lyapunov proof for SIR
The function $L(S,I) = S + I$ satisfies $L(S,I) \geq 0$ and is a candidate for a strict Lyapunov function.
Further only values in $[0,1]$ are considered for $S$ and $I$ as only those values have biological interpretations for the model (i.e ${\mathcal S} = [0,1]$).
\begin{align*}
    \nabla L &= (1,1)\\
    \nabla L \cdot X' &= (1,1)\cdot(-{\mathcal R_0}SI,{\mathcal R_0}SI-I)\\
    &=-{\mathcal R_0}SI + {\mathcal R_0}SI-I\\
    &=-I
\end{align*}
Since $\nabla L \cdot X' = -I < 0\quad \forall (S,I) \in \mathbb{R}^2\char`\\ \{(S,0)\quad \forall S \in \mathbb{R}\}$ all equilibria $(S,0)$ are asymptotically stable.
\end{enumerate}
%End
\centerline{\bf--- END OF ASSIGNMENT ---}
\bigskip
Compile time for this document:
\today\ @ \thistime
\end{document}
