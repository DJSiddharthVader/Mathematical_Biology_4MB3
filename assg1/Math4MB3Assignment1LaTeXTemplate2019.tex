\documentclass[12pt]{article}

\input{../preamble_for_assignments}
\input{4mba1q} % questions

%%%%%%%%%%%%%%%%%%%%%%%%%%%%%%%%%%%
%% FANCY HEADER AND FOOTER STUFF %%
%%%%%%%%%%%%%%%%%%%%%%%%%%%%%%%%%%%
\usepackage{fancyhdr,lastpage, dsfont}
\pagestyle{fancy}
\fancyhf{} % clear all header and footer parameters
%%%\lhead{Student Name: \theblank{4cm}}
%%%\chead{}
%%%\rhead{Student Number: \theblank{3cm}}
%%%\lfoot{\small\bfseries\ifnum\thepage<\pageref{LastPage}{CONTINUED\\on next page}\else{LAST PAGE}\fi}
\lfoot{}
\cfoot{{\small\bfseries Page \thepage\ of \pageref{LastPage}}}
\rfoot{}
\renewcommand\headrulewidth{0pt} % Removes funny header line
%%%%%%%%%%%%%%%%%%%%%%%%%%%%%%%%%%%

\begin{document}
%Title
\begin{center}
{\bfseries Mathematics 4MB3/6MB3 Mathematical Biology\\
\smallskip
2019 ASSIGNMENT {\color{blue}1}}\\
\medskip
\underline{\emph{Group Name}}: \texttt{{\color{blue}The Plague Doctors}}\\
\medskip
\underline{\emph{Group Members}}: {\color{blue}Sid Reed, Daniel Segura, Jessa Mallare, Aref Jadda}
\end{center}
\section{Analysis of the SI model}

\SIanalIntro
\begin{enumerate}[(a)]
\item \SIanalQa

  {\color{blue}
    \begin{proof}
      $ $\newline	
      Consider the open set $D: \{I\in \mathds{R}| 0 < I < N+\epsilon \}$ containing the the endemic equilibrium (EE) $I_* = N$ of        equation~\eqref{E:SI}, where $\epsilon > 0$ is constant. 
      Suppose a continuous differentiable function $L:D\rightarrow \mathds{R}$  is in the form 
      \begin{equation}
      L(I) = a(N-I)^b
      \end{equation} where $a$ and $b$ are constants and $a, b > 0$.\\
      
      Then by Lyapunov's Direct Method,
      \begin{enumerate}[(a)]
      	\item $L(I_*) = a(N-N)^b = 0$ and $L(I)>0$ \, $\forall I\in D\backslash \{I_*\}$
      	\item $\dot{L}(I)=\frac{dL}{dI}\frac{dI}{dt} \\
      	=-ab(N-I)^{b-1}\beta I(N-I)\\
      	=-ab\beta I(N-I)^b$\\
      	Let $a = 1$ and $b = 2$. Assuming $\beta>0$, then\\
      	$\dot{L}(I) = -2\beta I(N-I)^2 \leq 0$ \, $\forall I\in D\backslash \{I_*\}$
      	\item $\dot{L}(I) < 0$ \, $\forall I\in D\backslash \{I\}$
      \end{enumerate}
  	  Since the above (a), (b), and (c) hold true for $L(I) = (N-I)^2$, then by Lyapunov's Stability Theorem, $I_* = N$ is globally asymptotically stable. 
    \end{proof}
  }

\item \SIanalQb
  \begin{enumerate}[(i)]
  \item \SIanalQbi

    {\color{blue}
      \begin{proof}
        $ $\newline
      	$\frac{dI}{dt} = \beta I(N-I)$\\
      	$\int\frac{1}{I(N-I)} = \int\beta dt$\\
      	$\frac{1}{N}\int (\frac{1}{I} + \frac{1}{N-1})dI = \beta t + C$\\
      	$\frac{1}{N}(ln|I|- ln|N-I|) =\beta t + C$\\
      	$ln\abs{\frac{I}{N-I}} = N \beta t + NC$\\
      	$\frac{N-I}{I} = \pm e^{-NC}e^{-N\beta t}$\\
      	
      	Let $k = \pm e^{-NC}$\\
      	$\frac{N-I}{I} = ke^{-N\beta t}$\\
      	$\frac{N}{I} - 1 = ke^{-N\beta t}$\\
      	$\frac{N}{I} = 1 + ke^{-N\beta t}$\\
      	$I(t) = \frac{N}{1 + ke^{-N\beta t}}$\\      	
      	
      	To solve for $k$, let $I(0) = I_0$.\\
      	$I(0) = I_0 = \frac{N}{1 + k}$\\ 
      	$k = \frac{N - I_0}{I_0}$\\
      	
        So the exact solution of the model is
        \begin{equation}
          I(t) =  \frac{N}{1+\frac{N-I_0}{I_0}(e^{-\beta N t})}
        \end{equation}
        Since $$\lim_{t\to\infty} e^{-\beta N t} = 0$$
       $$\lim_{t\to\infty} I(t) = \frac{N}{1 +\frac{N-I_0}{I_0}(0)} = N$$
       then every solution that starts in the interval $(0, N)$ converges to the EE. 
      \end{proof}
    }

  \item \SIanalQbii

    {\color{blue}
      \begin{proof}
        {\color{magenta}\dots beautifully clear and concise text to be inserted here\dots}
      \end{proof}
    }

  \end{enumerate}
\end{enumerate}

\section{Analysis of the basic SIR  model}

\basicSIRanalIntro
\begin{enumerate}[(a)]
\item \basicSIRanalQa

{\color{blue}
\begin{proof}[Solution]
{\color{magenta}\dots beautifully clear and concise text to be inserted here\dots}
\end{proof}
}

\item \basicSIRanalQb
  \begin{enumerate}[(i)]
  \item \basicSIRanalQbi

{\color{blue}
\begin{proof}[Solution]
{\color{magenta}\dots beautifully clear and concise text to be inserted here\dots}
\end{proof}
}

 \item \basicSIRanalQbii

{\color{blue}
\begin{proof}[Solution]
{\color{magenta}\dots beautifully clear and concise text to be inserted here\dots}
\end{proof}
}

  \item \basicSIRanalQbiii
  \item \basicSIRanalQbiv
  \end{enumerate}
\item \basicSIRanalQc

{\color{blue}
\begin{proof}[Answers]
{\color{magenta}\dots beautifully clear and concise text to be inserted here\dots}
Since no numerical analysis or graphics were requested, there is no need to use \Rlogo to answer these questions.  Note, incidentally, that the \verb|\Rlogo| command that produces \Rlogo simply includes the image file \texttt{images/Rlogo.pdf} and scales it appropriately for text within a paragraph.  If we really loved that logo and wanted to display as we would a figure, we easily could\dots
\end{proof}
}
\begin{center}
\scalebox{3}{
\includegraphics{images/Rlogo.pdf}
}
\end{center}

\item \basicSIRanalQd

{\color{blue}
\begin{proof}[Solution]
{\color{magenta}\dots beautifully clear and concise text to be inserted here\dots}
\end{proof}
}

\end{enumerate}

\newpage
\hypertarget{NotesLyapFuns}{}
\section*{Notes on Lyapunov functions}

\NotesOnLyapunovFunctions

\bibliographystyle{vancouver}
\bibliography{4mba1_2019}

\bigskip

\centerline{\bf--- END OF ASSIGNMENT ---}

\bigskip
Compile time for this document:
\today\ @ \thistime

\end{document}
