I would advise my assistant to do this as it should be a fairly simple comparison and may provide insight in to the validity of the model.
Peak prevelance of the time series can be compared to the peak prevelance estimated from the initial conditions $(S_0,I_0)$ of the time series data.
If the model's prediction match the time series data very closely that may be indicative of some of the model's assumptions are better suited to modelling deaths as opposed to infections.\par
Although if deaths are assumed to match infections, then the model closely matching the time series data then that bodes well for the model.
That is not a necessairily a reasonable assumption as records about death are necessairily incomplete and not all deaths are attributable to disease, even with a perfect model the data will not match the predictions.
This discordance will mean that no matter what there will be uncertainty in  assessing the accuracy of the model using the time series data, but it still may  be a good qualitative indicator of the model.
