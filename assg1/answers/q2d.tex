All points $(S,0) S \in 0 \leq S \leq 1$ are equilibria for the SIR model.
The jacobian of the system is
$$ DF_{(S,I)} =
\begin{bmatrix}
    -{\mathcal R_0}I & -{\mathcal R_0}S\\
     {\mathcal R_0}I &  {\mathcal R_0}S - 1\\
\end{bmatrix} $$
Substituting in the equilibrium point gives:
$$ DF_{(S,0)} =
\begin{bmatrix}
    0 & -{\mathcal R_0}S\\
    0 &  {\mathcal R_0}S - 1\\
\end{bmatrix} $$
The eigenvalues of the matrix are the roots of the equation $\lambda^2 - T\lambda + D$ where $T$ and $D$ are the trace and determinant respectively.
\begin{align*}
    T &= {\mathcal R_0}S - 1\\
    D &= 0\\
    0 &= \lambda^2 - ({\mathcal R_0}S - 1)\lambda + (0)\\
    0 &= \lambda({\mathcal R_0}S - 1 + \lambda)\\\\
    \lambda &= 0\qquad or\qquad \lambda &= -({\mathcal R_0}S - 1)
\end{align*}
For one of the eigenvalues $R(\lambda) = 0$, therefore the equilibria are non-hyperbolic and the stability must be assessed in some other way.
%Lyapunov explanation
This assessment can be done by examining a Lyapunov function.
Lyapunov's theorem states that foran equilibrium point $X_*$ of $X'=F(X)$ and some set ${\mathcal S}$ if $\exists L(X)$ such that
\begin{align*}
    L(X_*) &= 0\\
    L(X) &\geq 0\quad \forall X \in S\char`\\ \{X_*\}\\
    \nabla L(X) \cdot X' &< 0 \quad \forall X \in S\char`\\ \{X_*\}
\end{align*}
then $L(X)$ is a strict Lyapunov function and $X_*$ is asymptotically stable.
%Lyapunov proof for SIR
The function $L(S,I) = S + I$ satisfies $L(S,I) \geq 0$ and is a candidate for a strict Lyapunov function.
Further only values in $[0,1]$ are considered for $S$ and $I$ as only those values have biological interpretations for the model (i.e ${\mathcal S} = [0,1]$).
\begin{align*}
    \nabla L &= (1,1)\\
    \nabla L \cdot X' &= (1,1)\cdot(-{\mathcal R_0}SI,{\mathcal R_0}SI-I)\\
    &=-{\mathcal R_0}SI + {\mathcal R_0}SI-I\\
    &=-I
\end{align*}
Since $\nabla L \cdot X' = -I < 0\quad \forall (S,I) \in \mathbb{R}^2\char`\\ \{(S,0)\quad \forall S \in \mathbb{R}\}$ all equilibria $(S,0)$ are asymptotically stable by Lyapunov's theorem.
