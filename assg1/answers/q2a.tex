The peak prevalence of the disease will be the maximum number of people infected at a given time $t$ during the infection.
%proof
This implies that the peak prevalence is the maximum value of the function $I(t)$.
In order to get a function of $I$ with the initial conditions $(S_0,I_0)$ as parameters we can consider the phase portrait solution:
$$ I_{max} = I_0 + S_0 - \frac{1}{\mathcal R_0} + \frac{1}{\mathcal R_0}log(\frac{1}{S_0\mathcal R_0}))$$
which can be derived as follows:
%I(S) derivation
\begin{align*}
    \frac{dI}{dS}     &= \frac{dI/dt}{dS/dt}\\
                      &= -1 + \frac{1}{S\mathcal R_0}\\
    \int_{I_0}^{I} dI &= \int_{S_0}^{S} -1 + \frac{1}{S\mathcal R_0}dS\\
    I - I_0           &= -(S - S_0) + \frac{1}{\mathcal R_0}log(\frac{S}{S_0})\\
    I                 &= I_0 + S_0 - S + \frac{1}{\mathcal R_0}log(\frac{S}{S_0})
\end{align*}
%max val of I(S)
All maxima and minima of a function occur at points $(x,y)$ such that $f'(x) = 0$, thus
\begin{align*}
    \frac{dI/dt}{dS/dt} &= \frac{{\mathcal R_0}SI - I}{-\mathcal R_0SI}\\
    \frac{dI/dt}{dS/dt} &= -1 + \frac{1}{S\mathcal R_0}\\
    0                   &= -1 + \frac{1}{S\mathcal R_0}\\
    1                   &= \frac{1}{S\mathcal R_0}\\
    1                   &= S{\mathcal R_0}
\end{align*}
This equation is true when $S = \frac{1}{\mathcal R_0}$, thus the maximum value for the function $I(S)$ occurs when $S = \frac{1}{\mathcal R_0}$.
Substituting this into $I(S)$ will give an expression for $I_{max}$ in terms of the initial conditions $(S_0,I_0)$
$$ I_{max} = I_0 + S_0 - \frac{1}{\mathcal R_0} + \frac{1}{\mathcal R_0}log(\frac{1}{S_0\mathcal R_0}))$$
%Justification
This quantity may be important to a public health officials for triage. If an epidemic is expected to have a low peak prevalence, fewer health-related resources would need to be allocated to treatment and prevention.
If the peak prevalence is estimated to be high then more effort may be directed towards prevention and treatment of the disease.
Further if the peak prevalence is estimated the time of peak prevalence can be derived easily using the $I(t)$ function, to estimate how much time exists to prepare for the peak of the infection, when resources (money, health personnel, equipment, etc.) will be most strained.
