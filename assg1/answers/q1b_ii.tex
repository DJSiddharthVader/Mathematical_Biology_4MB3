\begin{document}

When $I(t) \in (0,N)$,  $\frac{dI}{dt} >0\ $ since $\beta >0, I >0$, and $(N-I) >0$. 
Hence the function monotonically increases in that interval and is always moving away from 0 and goes towards N (the Endemic Equilibrium). 
N is also the mathematical (and biological) upper bound of the system of relevant solutions. 
Therefore, by the Monotone Convergence Theorem all solutions with the intial conditions $I(0) \in (0,N)$ converge to N as $t\to\infty$, and so N is an asymptotically stable equilibrium. 
Hence, given $\epsilon>0$, $\exists t<\infty$ such that $I(t) \in [N-\epsilon, N)$ for any $I(0) \in (0,N)$.\\
Considering that 'global' here only refers to biologically relevant conditions, only the range of $I(t) \in (0,N)$ needs to be analysed. 
Ergo, the Endemic Equilibrium (N) is globally asymptotically stable.

\end{document}
