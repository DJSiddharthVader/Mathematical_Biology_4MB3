In this case the susceptible population can come into contact with individuals in 4 different stages of infectiousness. Each $i$th term in $\mathcal R_0$ corresponds to the number of new infectious individuals per individual that stays in compartment $i$. For example, the first term in $\mathcal R_0$ correspond to the number of secondary infectious individuals per individual in the prodrom stage of the disease.\par
More specifically, each term in $\mathcal R_0$ is the product of the transmission rate of an individual in stage $i$ (each $\beta_i$ term), the proportion of individuals that survives to stage $i$, and the average time each individual that enters the $i$th stage stays in the $i$th stage.\par
For example, the second term in $\mathcal R_0$ corresponds the extreme infectivity stage, where the transmission rate of an individual in this stage is $\beta_e$, the proportion of individuals that survives to and enters this stage is $\frac{\sigma_1 \sigma_2}{(\sigma_1 + \mu)(\sigma_2 + \mu)}$, and the average time each individual that enters the extreme stage stays in this stage is $\frac{1}{\sigma_3 + \mu}$. \par
Ergo, each transmission possibility with the infectious individuals in all 4 stages should be calculated similar to the $SEIR$ model, and they can be summed up to return the $R_0$ value. Putting the 4 equations together we get:
\begin{align*}
&R_0 = \frac{\beta_r \sigma_{1}}{{\left(\sigma_{1} + \mu\right)} {\left(\sigma_{2} + \mu\right)}} + \frac{\beta_e \sigma_{1} \sigma_{2}}{{\left(\sigma_{1} + \mu\right)} {\left(\sigma_{2} + \mu\right)} {\left(\sigma_{3} + \mu\right)}}\\ 
&+ \frac{\beta_m \sigma_{1} \sigma_{2} \sigma_{3}}{{\left(\sigma_{1} + \mu\right)} {\left(\sigma_{2} + \mu\right)} {\left(\sigma_{3} + \mu\right)} {\left(\sigma_{4} + \mu\right)}} + \frac{\beta_l \sigma_{1} \sigma_{2} \sigma_{3} \sigma_{4}}{{\left(\sigma_{1} + \mu\right)} {\left(\sigma_{2} + \mu\right)} {\left(\sigma_{3} + \mu\right)} {\left(\sigma_{4} + \mu\right)} {\left(\mu + \gamma\right)}}
\end{align*}
