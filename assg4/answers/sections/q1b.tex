The multi-planel figure is a diagram of the time-series, paired with a wavelet power spectrum and three period periodograms of time time series. The time points at which the period periodograms were set to begin and end were chosen based on shifts in pattern of the time series as well as dampening of signal power of the wavelet diagram. From the wavelet diagram and the periodperiodograms of the london series, starting at 1944 there is a strong signal at $104$ and at $52$. As the data is given in weeks, a periodicity of $104$ weeks$=$2 $years and a signal at $52$ weeks $=1$ year. There is a  shift in the wavelet diagram ater the year 1970, as the power of the signal is for the biannual and annual cycles is much lower, but are still significant, as illustrated by the white contour lines of the wavelet diagram. THis drop in power can be attributed to the decrease ofin magnitudes of the peaks  in the time series diagram at this point in time. Comparing the periodogram from before 1970 to after, it is clear that there is much less pronounced periodicity, with only a small residual peak of the biannual cycle. Furthermore, after the year 1990, the power is no longer significant signal, meaning that there is no longer a significant frequency at the $2$  and $1$ year cycles. Due to this fact, we can attribute the peaks seen in the period periodogram as mostly noise for this period in time.  For the liverpool time series we see a similarl pattern as the one seen in london,  with a strong $2$ year and $1$ year signal from 1944 until 1965, at which point the signal begins to dampene out, as seen in both the wavelelt diagram and the decrease in magnitude of the peaks in the time series plot. The period periodogram of the time series from 1965-1980 shows peaks for the biannual and annual cycles that are much weaker in power than the peaks of th eperiodogram from 1944-1965. After 1980 the time series dampenes down, and the power of the signal of the period periodogram is no longer significant, similarly to what occurs in the london time series in the year 1990. Some interesting things about the figures are that the the signal of the liverpool data is not as strong as the signal of the london data, as the scale which goes to the maximum power only goes up to 2.00, where as the power scale for the london diagram goes up to 2.3. Similarly, the scale of power of the period periodogram of the london time series goes up to $8*10^7$, while it only goes up to $8*10^5$ for the liver pool time series, showing that the signal is allot stronger for the london data. Another interesting thing is how the wavelet diagrams look very similar for both london and liverpool, showing that the disease is functioning similarly at two different population levels. 
