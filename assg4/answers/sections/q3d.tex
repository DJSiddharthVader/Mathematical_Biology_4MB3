The early rash stage of the altered virus is expected to be twice as long as the original strain, meaning that the mean infectious period in the early rash stage ($\frac{1}{\sigma_3}$) has doubled. In other words, $\sigma_3$ is divided in half. So the $R_0$ for the new strain can be written as: 
\begin{align*}
&R_0 = \frac{\beta_r \sigma_{1}}{{\left(\sigma_{1} + \mu\right)} {\left(\sigma_{2} + \mu\right)}} + \frac{\beta_e \sigma_{1} \sigma_{2}}{{\left(\sigma_{1} + \mu\right)} {\left(\sigma_{2} + \mu\right)} {\left((\sigma_{3}/2) + \mu\right)}}\\ 
&+ \frac{\beta_m \sigma_{1} \sigma_{2} (\sigma_{3}/2)}{{\left(\sigma_{1} + \mu\right)} {\left(\sigma_{2} + \mu\right)} {\left((\sigma_{3}/2) + \mu\right)} {\left(\sigma_{4} + \mu\right)}} + \frac{\beta_l \sigma_{1} \sigma_{2} (\sigma_{3}/2) \sigma_{4}}{{\left(\sigma_{1} + \mu\right)} {\left(\sigma_{2} + \mu\right)} {\left((\sigma_{3}/2) + \mu\right)} {\left(\sigma_{4} + \mu\right)} {\left(\mu + \gamma\right)}}
\end{align*}\\
Since $\beta_e$ is the highest infectiousness rate (extreme), this change in $\sigma_3$ will have the most significant impact on the second fraction of the equation, $\frac{\beta_e \sigma_{1} \sigma_{2}}{{\left(\sigma_{1} + \mu\right)} {\left(\sigma_{2} + \mu\right)} {\left((\sigma_{3}/2) + \mu\right)}}$, almost doubling its value. \\ 
The changes in the last 2 fractions in $R_0$ are very minor and can be ignored. Since natural death rate ($\mu$) is much smaller than $\sigma_3$, the changes in the numerator almost cancel out the changes in the denominator completely. Thus, $R_0$ can be expected to increase significantly in the new virus.