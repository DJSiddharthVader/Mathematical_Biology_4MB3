\documentclass[12pt]{article}\usepackage[]{graphicx}\usepackage[]{color}
%% maxwidth is the original width if it is less than linewidth
%% otherwise use linewidth (to make sure the graphics do not exceed the margin)
\makeatletter
\def\maxwidth{ %
  \ifdim\Gin@nat@width>\linewidth
    \linewidth
  \else
    \Gin@nat@width
  \fi
}
\makeatother

\definecolor{fgcolor}{rgb}{0.345, 0.345, 0.345}
\newcommand{\hlnum}[1]{\textcolor[rgb]{0.686,0.059,0.569}{#1}}%
\newcommand{\hlstr}[1]{\textcolor[rgb]{0.192,0.494,0.8}{#1}}%
\newcommand{\hlcom}[1]{\textcolor[rgb]{0.678,0.584,0.686}{\textit{#1}}}%
\newcommand{\hlopt}[1]{\textcolor[rgb]{0,0,0}{#1}}%
\newcommand{\hlstd}[1]{\textcolor[rgb]{0.345,0.345,0.345}{#1}}%
\newcommand{\hlkwa}[1]{\textcolor[rgb]{0.161,0.373,0.58}{\textbf{#1}}}%
\newcommand{\hlkwb}[1]{\textcolor[rgb]{0.69,0.353,0.396}{#1}}%
\newcommand{\hlkwc}[1]{\textcolor[rgb]{0.333,0.667,0.333}{#1}}%
\newcommand{\hlkwd}[1]{\textcolor[rgb]{0.737,0.353,0.396}{\textbf{#1}}}%
\let\hlipl\hlkwb

\usepackage{framed}
\makeatletter
\newenvironment{kframe}{%
 \def\at@end@of@kframe{}%
 \ifinner\ifhmode%
  \def\at@end@of@kframe{\end{minipage}}%
  \begin{minipage}{\columnwidth}%
 \fi\fi%
 \def\FrameCommand##1{\hskip\@totalleftmargin \hskip-\fboxsep
 \colorbox{shadecolor}{##1}\hskip-\fboxsep
     % There is no \\@totalrightmargin, so:
     \hskip-\linewidth \hskip-\@totalleftmargin \hskip\columnwidth}%
 \MakeFramed {\advance\hsize-\width
   \@totalleftmargin\z@ \linewidth\hsize
   \@setminipage}}%
 {\par\unskip\endMakeFramed%
 \at@end@of@kframe}
\makeatother

\definecolor{shadecolor}{rgb}{.97, .97, .97}
\definecolor{messagecolor}{rgb}{0, 0, 0}
\definecolor{warningcolor}{rgb}{1, 0, 1}
\definecolor{errorcolor}{rgb}{1, 0, 0}
\newenvironment{knitrout}{}{} % an empty environment to be redefined in TeX

\usepackage{alltt}
\input{preamble}
\input{4mba4q}
%% FANCY HEADER AND FOOTER STUFF %%
\usepackage{fancyhdr,lastpage,tikz}
\usetikzlibrary{shapes,arrows}
\pagestyle{fancy}
\fancyhf{} % clear all header and footer parameters
%%%\lhead{Student Name: \theblank{4cm}}
%%%\chead{}
%%%\rhead{Student Number: \theblank{3cm}}
%%%\lfoot{\small\bfseries\ifnum\thepage<\pageref{LastPage}{CONTINUED\\on next page}\else{LAST PAGE}\fi}
\lfoot{}
\cfoot{{\small\bfseries Page \thepage\ of \pageref{LastPage}}}
\rfoot{}
\renewcommand\headrulewidth{0pt} % Removes funny header line
\IfFileExists{upquote.sty}{\usepackage{upquote}}{}
\begin{document}
%Title
\begin{center}
{\bf Mathematics 4MB3/6MB3 Mathematical Biology\\
\smallskip
\url{http://www.math.mcmaster.ca/earn/4MB3}\\
\smallskip
2019 ASSIGNMENT 4}\\
\medskip
\underline{\emph{Group Name}}: \texttt{{\color{blue}The Plague Doctors}}\\
\medskip
\underline{\emph{Group Members}}: {\color{blue}Sid Reed, Daniel Segura, Jessa Mallare, Aref Jadda}
\end{center}

%due date
\bigskip
\noindent
This assignment is {\bfseries\color{red} due in class} on \textcolor{red}{\bf Wednesday March 13 2019 at 10:30am}.
\bigskip

%Setting seed globally
\begin{knitrout}
\definecolor{shadecolor}{rgb}{0.969, 0.969, 0.969}\color{fgcolor}\begin{kframe}
\begin{alltt}
\hlcom{#Setting the seed for the entire document, for reproducible stochastic simulations}
\hlstd{seed} \hlkwb{=} \hlnum{9}
\hlkwd{set.seed}\hlstd{(seed)}
\end{alltt}
\end{kframe}
\end{knitrout}
%Answers
\begin{enumerate}
    \item %TSintro
    \begin{enumerate}[(a)]
        \item \TSa
        \begin{enumerate}[(i)]
            \item

% !Rnw root = main.Rnw
\begin{knitrout}
\definecolor{shadecolor}{rgb}{0.969, 0.969, 0.969}\color{fgcolor}\begin{kframe}
\begin{verbatim}
## [1] "First 5 rows of the dataframe"
##   year month day cases       date
## 1 1944     1   7    82 1944-01-07
## 2 1944     1  14    98 1944-01-14
## 3 1944     1  21   118 1944-01-21
## 4 1944     1  28   153 1944-01-28
## 5 1944     2   4   206 1944-02-04
## 6 1944     2  11   217 1944-02-11
\end{verbatim}
\end{kframe}
\end{knitrout}

            \item

% !Rnw root = main.Rnw
\begin{knitrout}
\definecolor{shadecolor}{rgb}{0.969, 0.969, 0.969}\color{fgcolor}\begin{kframe}
\begin{verbatim}
## [1] "Smoothing data with window size 30"
\end{verbatim}
\end{kframe}
\includegraphics[width=\maxwidth]{figure/1aii-1} 

\end{knitrout}
            \item

% !Rnw root = main.Rnw
\begin{knitrout}
\definecolor{shadecolor}{rgb}{0.969, 0.969, 0.969}\color{fgcolor}\begin{kframe}
\begin{verbatim}
## [1] "Using pgram estimation method"
\end{verbatim}
\end{kframe}
\includegraphics[width=\maxwidth]{figure/1aiii-1} 

\end{knitrout}


        \end{enumerate}
        \item \TSb

% !Rnw root = main.Rnw
\begin{center}
{\Large London Influenza Time Series}
\end{center}
\begin{knitrout}
\definecolor{shadecolor}{rgb}{0.969, 0.969, 0.969}\color{fgcolor}

{\centering \includegraphics[width=\maxwidth]{figure/London-1} 

}



\end{knitrout}
\newpage
\begin{center}
{\Large Liverpool Influenza Time Series}
\end{center}
\begin{knitrout}
\definecolor{shadecolor}{rgb}{0.969, 0.969, 0.969}\color{fgcolor}

{\centering \includegraphics[width=\maxwidth]{figure/Liverpool-1} 

}



\end{knitrout}
{\Large Describing The Inflenza Data}\\\\
\hspace{0.4in}The multi-planel figure is a diagram of the time-series, paired with a wavelet power spectrum and multiple period periodograms of time time series. The time points at which the period periodograms were set to begin and end were chosen based on shifts in pattern of the time series as well as dampening of signal power of the wavelet diagram. From the wavelet diagram and the periodperiodograms of the london series, starting at 1944 there is a strong signal at $104$ and at $52$. As the data is given in weeks, a periodicity of $104$ weeks$=$2 years and a signal at $52$ weeks $=1$ year. There is a  shift in the wavelet diagram ater the year 1970, as the power of the signal is for the biannual and annual cycles is much lower, but are still significant, as illustrated by the white contour lines of the wavelet diagram. THis drop in power can be attributed to the decrease ofin magnitudes of the peaks  in the time series diagram at this point in time. Comparing the periodogram from before 1970 to after, it is clear that there is much less pronounced periodicity, with only a small residual peak of the biannual cycle. Furthermore, after the year 1990, the power is no longer significant signal, meaning that there is no longer a significant frequency at the $2$  and $1$ year cycles. Which can be futher attributed the lack of peaks seen in the period periodogram for this period in time.\par


\hspace{0.4in}For the liverpool time series we see a similar pattern as the one seen in london,  with a strong $2$ year and $1$ year signal from 1944 until 1965 seen in the wavelet diagram and the period periodogram for this time. After 1965 the signal begins to dampen out, as seen in both the wavelelt diagram by the decrease in power of the signal,the time series plot by the decrease in magnitude of the peaks and in the period periodogram of 1965-1977 showing a small peak for the biannual and annual cycles, which are much weaker in power than the peaks of the periodogram from 1944-1965. After 1965 until 1977  the time series begins to dampene down futher, the power of the signal of the wavelet diagram is no longer significant, and the period periodogram of 1997-1990 shows no peaks. This means that there is no particular periodicity of significance for this time period. Similarly to what occurs in the london time series in the year 1990, the time series of the liverpool data dampens down even more, and both the wavelet diagram and the period periodogram are the same as in the 1977-1999 time period, showing that once again, there is no significant period. Some interesting things about the figures are that the the signal of the liverpool data is not as strong as the signal of the london data, as the scale which goes to the maximum power only goes up to 2.00, where as the power scale for the london diagram goes up to 2.3. Similarly, the scale of power of the period periodogram of the london time series goes up to $8*10^7$, while it only goes up to $8*10^5$ for the liver pool time series, showing that the signal is allot stronger for the london data.


\hspace{0.4in}\hspace{0.4in}Another interesting thing is how the wavelet diagrams look very similar for both london and liverpool, showing that the disease is functioning similarly at two different population levels. Some things that are puzzling about the figure are why the time series begins to dampen down in liverpool in 1965 where as the dampening begins to ocur in london in 1970. Also, it is of interest that  the time series of london begins to dampen right after the largest peak of the time series. Further, we see in the liverpool, in the third segment periodogram there appears to be very little power in that segment, despite the apperance of some oscillation in the time series. This may be an issue of the axis scales, as more clear difference might be observed with a smaller axis, but it is a clear contrast between the observed time series and the period periodograms.
    \end{enumerate}
    \item \SEintro
    \begin{enumerate}[(a)]
        \item

% !Rnw root = main.Rnw
\begin{knitrout}
\definecolor{shadecolor}{rgb}{0.969, 0.969, 0.969}\color{fgcolor}\begin{kframe}
\begin{alltt}
\hlkwd{source}\hlstd{(}\hlstr{'gillespie.R'}\hlstd{)}
\hlcom{#simulation vars}
\hlstd{beta} \hlkwb{=} \hlnum{1}
\hlstd{N} \hlkwb{=} \hlnum{10000}
\hlstd{I0} \hlkwb{=} \hlnum{1}
\hlstd{tmax} \hlkwb{=} \hlnum{20}
\hlstd{realizations} \hlkwb{=} \hlnum{30}
\hlstd{result} \hlkwb{<-} \hlkwd{SI.Gillespie}\hlstd{(beta,N,I0,tmax)}
\hlkwd{plot}\hlstd{(result[[}\hlnum{1}\hlstd{]], result[[}\hlnum{2}\hlstd{]],} \hlkwc{col}\hlstd{=}\hlstr{"green"}\hlstd{,} \hlkwc{type}\hlstd{=}\hlstr{"l"}\hlstd{,} \hlkwc{xlab}\hlstd{=}\hlstr{"Time (t)"}\hlstd{,}
     \hlkwc{ylab}\hlstd{=}\hlstr{"Incidence (I(t))"}\hlstd{,} \hlkwc{main}\hlstd{=}\hlkwd{paste}\hlstd{(}\hlstr{"N ="}\hlstd{,N))}
\end{alltt}
\end{kframe}
\includegraphics[width=\maxwidth]{figure/singlegillespie-1} 

\end{knitrout}
        \item

% !Rnw root = main.Rnw
\begin{knitrout}
\definecolor{shadecolor}{rgb}{0.969, 0.969, 0.969}\color{fgcolor}\begin{kframe}
\begin{alltt}
\hlkwd{source}\hlstd{(}\hlstr{'gillespie.R'}\hlstd{)}
\hlcom{#simulation vars}
\hlstd{beta} \hlkwb{=} \hlnum{1}
\hlstd{ns} \hlkwb{=} \hlkwd{c}\hlstd{(}\hlnum{32}\hlstd{,}\hlnum{100}\hlstd{,}\hlnum{1000}\hlstd{,}\hlnum{10000}\hlstd{)}
\hlstd{I0} \hlkwb{=} \hlnum{1}
\hlstd{tmax} \hlkwb{=} \hlnum{30}
\hlstd{realizations} \hlkwb{=} \hlnum{30}
\hlstd{colors} \hlkwb{<-} \hlkwd{colors}\hlstd{()}
\hlkwd{multipanel}\hlstd{(realizations,beta,ns,I0,tmax,}\hlkwc{colors}\hlstd{=colors)}
\end{alltt}
\end{kframe}
\includegraphics[width=\maxwidth]{figure/multipanelgillespie-1} 

\end{knitrout}
    \end{enumerate}
    \item \Rintro
    \smpxnathistfig
    \FloatBarrier
    \begin{enumerate}[(a)]
        \item \begin{align*}
&\frac{dS}{dt}= \nu - \beta_r I_r S- \beta_e I_e S- \beta_m I_m S- \beta_l I_l S - \mu S \\
&\frac{dE}{dt}= \beta_r I_r S+ \beta_e I_e S+ \beta_m I_m S+ \beta_l I_l S - \sigma_{1} E - \mu E\\
&\frac{dI_r}{dt}= \sigma_{1} E - \sigma_{2} I_r - \mu I_r\\
&\frac{dI_e}{dt}= \sigma_{2} I_r - \sigma_{3} I_e - \mu I_e\\
&\frac{dI_m}{dt}= \sigma_{3} I_e - \sigma_{4} I_m - \mu I_m\\
&\frac{dI_l}{dt}= \sigma_{4} I_m - \gamma I_l - \mu I_l\\
&\frac{dR}{dt}= \gamma I_l - \mu R\\
\end{align*}
        \item \begin{align*}
	R_0 = \frac{\beta_r \sigma_{1}}{{\left(\sigma_{1} + \mu\right)} {\left(\sigma_{2} + \mu\right)}} + \frac{\beta_e \sigma_{1} \sigma_{2}}{{\left(\sigma_{1} + \mu\right)} {\left(\sigma_{2} + \mu\right)} {\left(\sigma_{3} + \mu\right)}} + \frac{\beta_m \sigma_{1} \sigma_{2} \sigma_{3}}{{\left(\sigma_{1} + \mu\right)} {\left(\sigma_{2} + \mu\right)} {\left(\sigma_{3} + \mu\right)} {\left(\sigma_{4} + \mu\right)}} \\ 
	&+ \frac{\beta_l \sigma_{1} \sigma_{2} \sigma_{3} \sigma_{4}}{{\left(\sigma_{1} + \mu\right)} {\left(\sigma_{2} + \mu\right)} {\left(\sigma_{3} + \mu\right)} {\left(\sigma_{4} + \mu\right)} {\left(\mu + \gamma\right)}}
\end{align*}
        \item Assuming that $\mathcal{F} =$ inflow of new infecteds to infected compartments, and $\mathcal{V} =$ outflow from infected compartments minus inflow of non-new infecteds we have:
\begin{gather*}
\mathcal{F} = \left(\begin{array}{c}
\beta_r I_r S+ \beta_e I_e S+ \beta_m I_m S+ \beta_l I_l S\\
0 \\
0 \\
0 \\
0
\end{array}\right) \;\;
\mathcal{V} = \left(\begin{array}{c}
\sigma_1 E + \mu E\\
-\sigma_1 E + \sigma_2 I_r + \mu I_r \\
-\sigma_2 I_r + \sigma_3 I_e + \mu I_e \\
-\sigma_3 I_e + \sigma_4 I_m + \mu I_m \\
-\sigma_4 I_m + \gamma I_l + \mu I_l
\end{array}\right)
\end{gather*}\\
Let $F=$ linearization of $\mathcal{F}$ at DFE.\\
Let $V=$ linearization of $\mathcal{V}$ at DFE. Calculating F and V we have:
\begin{gather*}
F=\left(\begin{array}{rrrrr}
0 & \beta_r & \beta_e & \beta_m & \beta_l \\
0 & 0 & 0 & 0 & 0 \\
0 & 0 & 0 & 0 & 0 \\
0 & 0 & 0 & 0 & 0 \\
0 & 0 & 0 & 0 & 0
\end{array}\right)\\
V=\left(\begin{array}{rrrrr}
(\sigma_1 + \mu) & 0 & 0 & 0 & 0 \\
-\sigma_1 & (\sigma_2 + \mu) & 0 & 0 & 0 \\
0 & -\sigma_2 & (\sigma_3 + \mu) & 0 & 0 \\
0 & 0 & -\sigma_3 & (\sigma_4 + \mu) & 0 \\
0 & 0 & 0 & -\sigma_4 & (\gamma + \mu)
\end{array}\right)
\end{gather*}\\
The next generation matrix is $FV^{-1}$. Calculating $FV^{-1}$ using a symbolic manipulation software we get:
\begin{align*}
&FV^{-1}= \left(\begin{array}{rrrrr}
m_1 & m_2 & m_3 & m_4 & m_5 \\
0 & 0 & 0 & 0 & 0 \\
0 & 0 & 0 & 0 & 0 \\
0 & 0 & 0 & 0 & 0 \\
0 & 0 & 0 & 0 & 0
\end{array}\right)\\
\end{align*}
\begin{align*}
where \\
&m_1 = \frac{\beta_r \sigma_{1}}{{\left(\sigma_{1} + \mu\right)} {\left(\sigma_{2} + \mu\right)}} + \frac{\beta_e \sigma_{1} \sigma_{2}}{{\left(\sigma_{1} + \mu\right)} {\left(\sigma_{2} + \mu\right)} {\left(\sigma_{3} + \mu\right)}} \\ 
&+ \frac{\beta_m \sigma_{1} \sigma_{2} \sigma_{3}}{{\left(\sigma_{1} + \mu\right)} {\left(\sigma_{2} + \mu\right)} {\left(\sigma_{3} + \mu\right)} {\left(\sigma_{4} + \mu\right)}} + \frac{\beta_l \sigma_{1} \sigma_{2} \sigma_{3} \sigma_{4}}{{\left(\sigma_{1} + \mu\right)} {\left(\sigma_{2} + \mu\right)} {\left(\sigma_{3} + \mu\right)} {\left(\sigma_{4} + \mu\right)} {\left(\mu + \gamma\right)}} \\
&m_2= \frac{\beta_r}{\sigma_{2} + \mu} + \frac{\beta_e \sigma_{2}}{{\left(\sigma_{2} + \mu\right)} {\left(\sigma_{3} + \mu\right)}} + \frac{\beta_m \sigma_{2} \sigma_{3}}{{\left(\sigma_{2} + \mu\right)} {\left(\sigma_{3} + \mu\right)} {\left(\sigma_{4} + \mu\right)}} \\ & + \frac{\beta_l \sigma_{2} \sigma_{3} \sigma_{4}}{{\left(\sigma_{2} + \mu\right)} {\left(\sigma_{3} + \mu\right)} {\left(\sigma_{4} + \mu\right)} {\left(\mu + \gamma\right)}} \\
&m_3= \frac{\beta_e}{\sigma_{3} + \mu} + \frac{\beta_m \sigma_{3}}{{\left(\sigma_{3} + \mu\right)} {\left(\sigma_{4} + \mu\right)}} + \frac{\beta_l \sigma_{3} \sigma_{4}}{{\left(\sigma_{3} + \mu\right)} {\left(\sigma_{4} + \mu\right)} {\left(\mu + \gamma\right)}} \\
&m_4= \frac{\beta_m}{\sigma_{4} + \mu} + \frac{\beta_l \sigma_{4}}{{\left(\sigma_{4} + \mu\right)} {\left(\mu + \gamma\right)}} \\
&m_5= \frac{\beta_l}{\mu + \gamma} \\
\end{align*}
$R_0$ can be calculated as the spectral radius of the matrix $FV^{-1}$ (or $\rho(FV^{-1})$). Therefore:
\begin{align*}
&R_0 = \rho(FV^{-1}) = \frac{\beta_r \sigma_{1}}{{\left(\sigma_{1} + \mu\right)} {\left(\sigma_{2} + \mu\right)}} + \frac{\beta_e \sigma_{1} \sigma_{2}}{{\left(\sigma_{1} + \mu\right)} {\left(\sigma_{2} + \mu\right)} {\left(\sigma_{3} + \mu\right)}} \\ 
&+ \frac{\beta_m \sigma_{1} \sigma_{2} \sigma_{3}}{{\left(\sigma_{1} + \mu\right)} {\left(\sigma_{2} + \mu\right)} {\left(\sigma_{3} + \mu\right)} {\left(\sigma_{4} + \mu\right)}} + \frac{\beta_l \sigma_{1} \sigma_{2} \sigma_{3} \sigma_{4}}{{\left(\sigma_{1} + \mu\right)} {\left(\sigma_{2} + \mu\right)} {\left(\sigma_{3} + \mu\right)} {\left(\sigma_{4} + \mu\right)} {\left(\mu + \gamma\right)}}
\end{align*}
        \item The early rash stage of the altered virus is expected to be twice as long as the original strain, meaning that the mean infectious period in the early rash stage ($\frac{1}{\sigma_3}$) has doubled. In other words, $\sigma_3$ is divided in half. So the $R_0$ for the new strain can be written as: 
\begin{align*}
&R_0 = \frac{\beta_r \sigma_{1}}{{\left(\sigma_{1} + \mu\right)} {\left(\sigma_{2} + \mu\right)}} + \frac{\beta_e \sigma_{1} \sigma_{2}}{{\left(\sigma_{1} + \mu\right)} {\left(\sigma_{2} + \mu\right)} {\left((\sigma_{3}/2) + \mu\right)}}\\ 
&+ \frac{\beta_m \sigma_{1} \sigma_{2} (\sigma_{3}/2)}{{\left(\sigma_{1} + \mu\right)} {\left(\sigma_{2} + \mu\right)} {\left((\sigma_{3}/2) + \mu\right)} {\left(\sigma_{4} + \mu\right)}} + \frac{\beta_l \sigma_{1} \sigma_{2} (\sigma_{3}/2) \sigma_{4}}{{\left(\sigma_{1} + \mu\right)} {\left(\sigma_{2} + \mu\right)} {\left((\sigma_{3}/2) + \mu\right)} {\left(\sigma_{4} + \mu\right)} {\left(\mu + \gamma\right)}}
\end{align*}\\
Since $\beta_e$ is the highest infectiousness rate (extreme), this change in $\sigma_3$ will have the most significant impact on the second fraction of the equation, $\frac{\beta_e \sigma_{1} \sigma_{2}}{{\left(\sigma_{1} + \mu\right)} {\left(\sigma_{2} + \mu\right)} {\left((\sigma_{3}/2) + \mu\right)}}$, almost doubling its value. \\ 
The changes in the last 2 fractions in $R_0$ are very minor and can be ignored. Since natural death rate ($\mu$) is much smaller than $\sigma_3$, the changes in the numerator almost cancel out the changes in the denominator completely. Thus, $R_0$ can be expected to increase significantly in the new virus.
        \item If the bio-terrorists were able to double the early rash stage of smallpox, this will greatly increase the reproductive ratio $\mathcal R_0$ by nearly doubling what was calculated for the unaltered strain of the disease ($\mathcal R_0 \simeq 5$). Increasing $\mathcal R_0$ will increase the critical proportion of vaccinated individuals in the population from $75\%$ to $95\%$ in order to prevent an epidemic. Without intervention, the final size of an  epidemic caused by the unaltered strain is estimated to be $95\%$. 	With the altered strain, the final size of the epidemic without any intervention will be similar but much closer to $100\%$. Thus it is imperative to implement vaccination, isolation, and quarantine intervention methods as soon as possible.

    \end{enumerate}
\end{enumerate}
%End Note
\bigskip\vfill
\centerline{\bf--- END OF ASSIGNMENT ---}
\bigskip
Compile time for this document:
\today\ @ \thistime
\end{document}
