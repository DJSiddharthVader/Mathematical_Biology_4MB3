%Preamble
\documentclass[12pt]{article}\usepackage[]{graphicx}\usepackage[]{color}
%% maxwidth is the original width if it is less than linewidth
%% otherwise use linewidth (to make sure the graphics do not exceed the margin)
\makeatletter
\def\maxwidth{ %
  \ifdim\Gin@nat@width>\linewidth
    \linewidth
  \else
    \Gin@nat@width
  \fi
}
\makeatother

\definecolor{fgcolor}{rgb}{0.345, 0.345, 0.345}
\newcommand{\hlnum}[1]{\textcolor[rgb]{0.686,0.059,0.569}{#1}}%
\newcommand{\hlstr}[1]{\textcolor[rgb]{0.192,0.494,0.8}{#1}}%
\newcommand{\hlcom}[1]{\textcolor[rgb]{0.678,0.584,0.686}{\textit{#1}}}%
\newcommand{\hlopt}[1]{\textcolor[rgb]{0,0,0}{#1}}%
\newcommand{\hlstd}[1]{\textcolor[rgb]{0.345,0.345,0.345}{#1}}%
\newcommand{\hlkwa}[1]{\textcolor[rgb]{0.161,0.373,0.58}{\textbf{#1}}}%
\newcommand{\hlkwb}[1]{\textcolor[rgb]{0.69,0.353,0.396}{#1}}%
\newcommand{\hlkwc}[1]{\textcolor[rgb]{0.333,0.667,0.333}{#1}}%
\newcommand{\hlkwd}[1]{\textcolor[rgb]{0.737,0.353,0.396}{\textbf{#1}}}%
\let\hlipl\hlkwb

\usepackage{framed}
\makeatletter
\newenvironment{kframe}{%
 \def\at@end@of@kframe{}%
 \ifinner\ifhmode%
  \def\at@end@of@kframe{\end{minipage}}%
  \begin{minipage}{\columnwidth}%
 \fi\fi%
 \def\FrameCommand##1{\hskip\@totalleftmargin \hskip-\fboxsep
 \colorbox{shadecolor}{##1}\hskip-\fboxsep
     % There is no \\@totalrightmargin, so:
     \hskip-\linewidth \hskip-\@totalleftmargin \hskip\columnwidth}%
 \MakeFramed {\advance\hsize-\width
   \@totalleftmargin\z@ \linewidth\hsize
   \@setminipage}}%
 {\par\unskip\endMakeFramed%
 \at@end@of@kframe}
\makeatother

\definecolor{shadecolor}{rgb}{.97, .97, .97}
\definecolor{messagecolor}{rgb}{0, 0, 0}
\definecolor{warningcolor}{rgb}{1, 0, 1}
\definecolor{errorcolor}{rgb}{1, 0, 0}
\newenvironment{knitrout}{}{} % an empty environment to be redefined in TeX

\usepackage{alltt}
\usepackage{scrtime} % for \thistime (this package MUST be listed first!)
\usepackage[margin=1in]{geometry}
\usepackage{graphics,graphicx}
\usepackage{placeins}
\usepackage[displaymath, mathlines]{lineno}
\usepackage{color}
\definecolor{aqua}{RGB}{0, 128, 225}
\usepackage[colorlinks=true,citecolor=aqua,linkcolor=aqua,urlcolor=aqua]{hyperref}
%For Math stuff
\usepackage{amsmath} % essential for cases environment
\usepackage{amsthm} % for theorems and proofs
\usepackage{amsfonts, array, siunitx} % mathbb
%Referenes
\usepackage[natbib=true,
            style=nature,
            citestyle=authoryear,
            backend=biber,
            useprefix=true]{biblatex}
\addbibresource{./plagueDoctors.bib}
%FANCY HEADER AND FOOTER STUFF
\usepackage{fancyhdr,lastpage}
\pagestyle{fancy}
\fancyhf{} % clear all header and footer parameters
%%%\lhead{Student Name: \theblank{4cm}}
%%%\chead{}
%%%\rhead{Student Number: \theblank{3cm}}
%%%\lfoot{\small\bfseries\ifnum\thepage<\pageref{LastPage}{CONTINUED\\on next page}\else{LAST PAGE}\fi}
\lfoot{}
\cfoot{{\small\bfseries Page \thepage\ of \pageref{LastPage}}}
\rfoot{}
\renewcommand\headrulewidth{0pt} % Removes funny header line
\IfFileExists{upquote.sty}{\usepackage{upquote}}{}
\begin{document}
%Title
\title{Examining Control Strategies for Cholera Incorporating Spatial Dynamics}
\author{
\underline{\emph{Group Name}}: \texttt{{\color{blue}The Plague Doctors}}\\\\
\underline{\emph{Group Members}}:\\
         Sid Reed\ :\ {\color{blue}reeds4@mcmaster.ca}\\
         Daniel Segura\ :\ {\color{blue}segurad@mcmaster.ca}\\
         Jessa Mallare\ :\ {\color{blue}mallarej@mcmaster.ca}\\
         Aref Jadda\ :\ {\color{blue}hossesa@mcmaster.ca}\\
}
\date{\today\ @ \thistime}
\maketitle
%due date
%\bigskip
%\noindent
%This assignment is {\bfseries\color{red} due in class} on \textcolor{red}{\bf Wednesday March 27 2019 at 10:30am}.
%\bigskip

\linenumbers

\begin{abstract}
    Cholera has a significant impact on public health, especially in areas with poor water sanitation, with infections estimated to affect between 1.3 and 4 million people annually \citep{link18}.
    Many people globally still lack proper infrastructure or access to clean water\citep{link19}.
    While vaccines and antibiotics exist, vaccination can be difficult to achieve at necessary levels for stopping an epidemic and widespread antibiotic usage contributes to the development of antibiotic resistance.
    Considering the spatial dynamics of water sanitation, antibiotic usage and vaccination are important for creating the most effective and efficient treatment regime in preventing cholera epidemics
\end{abstract}

\clearpage
\tableofcontents
\clearpage

%Sections
\section{Introduction}

\subsection{Biology of Cholera}
Although it is listed as one of the oldest known diseases, cholera remains a major public health concern in areas with poor water sanitation with an estimated 1.3-4 million cases every year \citep{link18}.
Cholera is an infectious disease caused by the bacterium \textit{Vibrio cholerae}.
The bacterium survives and reproduces in aquatic environments, and is capable of colonizing small intestines \citep{link8}.
The disease is not airborne, but can be transmitted through contaminated food or water and can survive in some aquatic environments from months to years \citep{link9}.
The bacterium produces enterotoxins responsible for the symptoms of cholera infection which are severe diarrhea, vomiting and nausea \citep{link11}.
%Approximately only one in ten people infected develop symptoms, and if not treated urgently, these symptoms can lead to severe dehydration.
Dehydration thickens the blood, causing circulation problems that can lead to death within a few hours.
Since dehydration is the main problem, rehydration with clean water and minerals (such as ORS packages) is the most effective treatment \citep{link18}.
Current improvements in public health and sanitation largely decrease the likelihood of a cholera outbreak \citep{link18}.\par
Four major outbreaks of cholera in the 19th century devastated the London population, resulting in tens of thousands of deaths.
One of the early theories believed to be the cause of spread of cholera was the Miasma theory, suggesting that cholera is an airborne disease and that impurities in the air induced the spread \citep{link1}.
Thus, the suggested solution in 1848 was to discard the contents of cesspools and raw sewage pits into the River Thames.
Since Thames was the drinking source of many, the misunderstanding about the method of transmission resulted in heightened number of infected individuals, severely worsening the epidemic \citep{link1}.
Early studies on cholera, such as the work of Jon Snow in the mid 19th century, have been pivotal in the development of modern epidemiology.
However, the abundance of more recent studies using mathematical frameworks that try to model a framework for anticipating outbreaks of cholera and planning for interventions is the reason for our focus on this particular disease.
\subsection{Transmission Dynamics of Cholera}
Before introducing a simple model to simulate the temporal spread of cholera, we must discuss the processes we plan to analyze.
The model should include the entire population, which for simplicity we will assume is comprised of only three groups: the susceptible, the infected (or infectious), and the recovered.
The only area still remaining that has a major impact on the epidemic is the environment, or in this case the water.
We assume that only Infectious individuals can contaminate the water sources by shedding the pathogen into the water.
The halting remedy suggested increased the rate of water contamination drastically, which in turn increased the transmission rate from individuals coming into contact with the infected water.
This is a plausible explanation for why maximum weekly deaths in London increase more than two-fold in the 1848 epidemic compared to the 1832 epidemic \citep{link3}.
The main treatment strategies for cholera outbreaks are vaccination, antibiotic treatment, and water sanitation.
We can incorporate these into our model to simulate the effect that each of these strategies has on the disease dynamics.
\subsection{SIRW Model Construction}
Our model has five distinct departments, a susceptible, exposed, infectious, removed, and water compartment.
Susceptible: contains the proportion of the population that is capable of becoming infected.
Individuals are born into compartment S at a rate of $\mu$.
Individuals leave the compartment in one of two ways, they die at a rate $\mu$, or come into contact with the pathogen and move into the “Infectious” compartment.
Interactions of susceptible and infected individuals from the I compartment yields new infected individuals at a rate of $\beta_I$, and interactions of susceptible individuals with the water compartment W yields new infected individuals at a rate $\beta_w$.\par
Infectious: Contains the proportion of individuals that enter from the S compartment in the manner discussed above.
Individuals in this compartment are capable of infecting susceptible individuals during interactions at a rate of $\beta_i$.
They are also capable of contributing to the choleric load of the water compartment by “shedding” the pathogen at a rate $\xi$.
Individuals in this compartment recover at a rate $\gamma$, and move to the R compartment, else they leave this compartment as they die (not from Cholera) at a rate $\mu$ and from Cholera at a rate $\alpha$.\par
Recovered: Contains the proportion of individuals that are neither infectious or susceptible to the pathogen.
They leave this compartment as they die at a rate $\mu$.
Water: The w term is proportional to the concentration of Cholera in the environment.
More bacteria enters the compartment as infected individuals “shed” the pathogen at a rate $\xi$, and the pathogen dies at a rate $\sigma$.
\section{Single Patch Models}
\subsection{Model Introduction And Parameters}

In this paper, we consider the cholera SIWR model as outlined by Tien and Earn \cite{link9} with the addition of death rate by cholera $\sigma$.  The tables below
summarize the variables and parameters involved.

%%VARIABLE Table%%
\begin{center}
	\begin{tabular}{ | m{1em} | m{8.14cm}| m{5.5cm} | }
		\hline
		\textbf{ }& \textbf{Description} & \textbf{Units} \\
		\hline
		S & Susceptible individuals & individuals \\
		\hline
		I & infected individuals & individuals \\
		\hline
		R & recovered individuals & individuals \\
		\hline
		W & Bacterial concentration in water & cells ml$^{-3}$ \\
		\hline
		N & Total number of individuals & individuals\\
		\hline
	\end{tabular}
\end{center}

%%PARAMETER Table%%
\begin{center}
	\begin{tabular}{ | m{1em} | m{8cm}| m{3cm} | m{2.2cm} | }
		\hline
		\textbf{ }& \textbf{Description} & \textbf{Units} &  \textbf{Estimate} \\
		\hline
		$\mu$ & Natural death/birth rate & day $^{-1}$ & \\
		\hline
		$b_i$ &  Person-person transmission/contact rate & cells ml$^{-3}$ day$^{-1}$ & \\
		\hline
		$b_w$ & water reservoir-person transmission/contact rate & cells ml$^{-3}$ day$^{-1}$&  \\
		\hline
		$\beta_i$ & scaled Person-person transmission/contact rate & day$^{-1}$ & 0.25\\
		\hline
		$\beta_w$ & scaled water reservoir-person transmission/contact rate & day$^{-1}$& \num{1e-5} to 1 \\
		\hline
		%$\kappa$ & Bacterial concentration in water & cells ml$^{-3}$ \\
		%\hline
		$\frac{1}{\gamma}$ & Infectious period & day& 2.9 to 14\\
		\hline
		$\sigma$ & Bacterial decay/removal from reservoir & day$^{-1}$& $\frac{1}{3}$ to $\frac{1}{41}$ \\
		\hline
		$\xi$ & Person to water reservoir shedding rate  & cells ml$^{-3}$ day$^{-1}$ individuals$^{-1}$ & 0.01 to 10\\
		\hline
		$\alpha$ & Death rate by cholera & day$^{-1}$& 0.01 to 0.6 \\
		\hline
	\end{tabular}
\end{center}

Parameter estimates are taken from \cite{link5}, \cite{link8} and \cite{link3}.
The natural death rate is dependent on various factors such as city or location and year or era of interest.

\subsection{Single Patch SIR Model With A Water Compartment}

\begin{linenomath}
	\begin{align*}
		\frac{dS}{dt}&= \mu N - \mu S - \beta_i SI - \beta_w S W  \\
		\frac{dI}{dt}&= \beta_i S I + \beta_w S W - I (\gamma + \mu + \alpha) \\
		\frac{dR}{dt}&= \gamma I - \mu R \\
		\frac{dW}{dt}&= \xi I  - \sigma W
	\end{align*}
\end{linenomath}
\begin{itemize}
    \item$\mu=$ natural death rate
    \item$\beta_i=$ transmission rate between S and I class
    \item$\beta_w=$ transmission rate between I and W class
    \item$\gamma=$ recovery rate (I to R class)
    \item$\alpha=$ death rate from cholera
    \item$\xi=$ Shedding rate of cholera from I to W class
    \item$\sigma=$	Removal rate of cholera from W class (depends on what we define as our water source)
\end{itemize}
The assumptions for this model are
\begin{itemize}
    \item The population is homogenously succeptible to Cholera infection
    \item No waning immunity; once you recover from cholera you cannot return to the susceptible class
    \item The transmission rate between water the susceptible class  is exponentially distributed
\end{itemize}

\begin{knitrout}
\definecolor{shadecolor}{rgb}{0.969, 0.969, 0.969}\color{fgcolor}\begin{figure}[h]
\includegraphics[width=\maxwidth]{figure/_singlePatch-1} \caption{\label{fig:singlepatch} Plot of the SIRW model for a single patch. Parameters are $\mu=0\ \beta_i=0.02\ \gamma=0.14\ \sigma=0.04\ \beta_w=0.5\ \alpha=0\ \xi=10$.Further the initial conditions for the model were $S_0=0.92\ I_0=0.08\ R_0=0$}\label{fig:<singlePatch}
\end{figure}


\end{knitrout}
\FloatBarrier
\subsection{Equilibrium and {$\mathcal R_0$} Of The Single Patch Model}

The basic reproductive number ${\mathcal R_0}$ is defined as the number of secondary infections as a result of a single infective during a time step.
${\mathcal R_0}$ can be computed as the spectral radius (i.e. the eigenvalue with the largest absolute value) of the next generation matrix at the disease free equilibrium.
The next generation matrix $FV^{−1}$, where the entry $F_{ij}$ of the matrix $F$ is the rate at which infected individuals in compartment $j$ produce new infections in compartment $i$, and the entry of $V_{ij}$ of the matrix $V$ is the mean time spent in compartment $j$ after moving into $j$ from compartment $k$.
For our model, we have
\begin{linenomath}
\begin{align*}
		F&=\begin{pmatrix}
			\beta_i & \beta_w\\
			0 & 0
			\end{pmatrix}\\
		V&=\begin{pmatrix}
			\frac{1}{\gamma+\mu+\alpha} & 0\\
			\frac{1}{\gamma+\mu+\alpha} &\frac{1}{\theta}
			\end{pmatrix}
\end{align*}
\end{linenomath}
The basic reproductive number is computed as the spectral radius of $FV^{-1}$ as seen in \cite{link9}, which is
\begin{linenomath}
\begin{align*}
    {\mathcal R_0} &= \rho(FV^{-1}\\
		           &=\frac{\beta_i+\beta_w}{\gamma+\mu}
\end{align*}
\end{linenomath}
This singla patch model has a disease-free equillibrium at (S,I,R)=(1,0,0) when ${\mathcal R_0}<1$.
It also has an endemic-equillirbium when ${\mathcal R_0}>1$

\subsection{Single Patch With Low And High Shedding Compartments}

%\textbf{Single Patch Model: Severity of Shedding dependent on Intensity of Symptoms (Low and High)}
\begin{linenomath}
\begin{align*}
	\frac{dS}{dt}&= \mu N - \mu S - \beta_L S I_L - \beta_H S I_H - \beta_w S W  \\
	\frac{dI_L}{dt}&= \beta_i S( I_L + I_H) + \beta_w S W - I_L (\mu + \delta + \alpha_L) \\
	\frac{dI_H}{dt}&= \delta I_L - I_H (\gamma + \mu + \alpha_H) \\
	\frac{dR}{dt}&= \gamma I_H - \mu R \\
	\frac{dW}{dt}&= \xi_L I_L + \xi_H I_H  - \sigma W\\
	\end{align*}
\end{linenomath}

This model assumes that you start off with low intensity symptoms (lower rate of shedding) and the symptoms reach a high intensity with a greater rate of shedding. The assumptions for the single patch model apply here as well.
\begin{itemize}
	\item$\alpha_i=$ death rate by cholera in low or high intensity
	\item$\delta =$ rate at which symptoms increase in severity
\end{itemize}

\section{Multi Patch Model}

%\section{Multi-Patch Models Of Cholera}
The following equations represent the SIRW model for a single patch $i$ in the multi patch model.
\begin{linenomath}
\begin{align*}
    \frac{dS_i}{dt}&= \mu N - \mu S_i - \beta_i S_i I_i - \phi \beta_i S_i \sum_j^n I_j - \beta_w S_i W_i - \psi \beta_w S_i \sum_j^n W_j\\
    \frac{dI_i}{dt}&= \beta_i S_i I_i + \beta_i \phi S_i \sum_j^n I_j + \beta_w S_i W_i + \beta_i \psi S_i \sum_j^n W_j - I_i (\gamma + \mu + \alpha) \\
    \frac{dR_i}{dt}&= \gamma I_i - \mu R_i \\
    \frac{dW_i}{dt}&= \xi I_i + \beta_i \psi I_i \sum_j^n W_j  - \sigma W_i
\end{align*}
\end{linenomath}
Where the set $n$ in the set is all neighbours (i.e. adjacent and directly diagonal patches) of the patch $i$.

\begin{itemize}
    \item$\mu=$ natural death rate
    \item$\phi=$ person to person contact rate between neighbouring patchs
    \item$\psi=$ person to water contact rate between neighbouring patchs
    \item$\beta_i=$ transmission rate between S and I class
    \item$\beta_w=$ transmission rate between I and W class
    \item$\gamma=$ recovery rate (I to R class)
    \item$\alpha=$ death rate from cholera
    \item$\xi=$ Shedding rate of cholera from I to W class
    \item$\sigma=$	Removal rate of cholera from W class (depends on what we define as our water source)
\end{itemize}
The assumptions for the single patch model apply here as well the following.
\begin{itemize}
    \item No dispersal of individuals between patches
    \item infected individuals in patch $i$ can infect succeptible individuals in the neihgbouring patch $j$
    \item All patches neighbouring $i$ have the same trasmission rate to patch $i$
\end{itemize}

\begin{knitrout}
\definecolor{shadecolor}{rgb}{0.969, 0.969, 0.969}\color{fgcolor}\begin{figure}
\includegraphics[width=\maxwidth]{figure/_multiPatch-1} \caption{\label{fig:multipatch} Plot of the SIRW model for all patches in a multi patch model. Parameters are $\mu=0\ \beta_i=0.02\ \gamma=0.14\ \sigma=0.04\ \beta_w=0.5\ \alpha=0\ \xi=10\ \phi=0.01\ \psi=0.05$. The initial conditions for the model were $S_0=0.94\ I_0=0.06\ R_0=0$.}\label{fig:<multiPatch}
\end{figure}


\end{knitrout}
\FloatBarrier
\section{Treatment Strategies For Cholera}

%\section{Possible Treatment Strategies for Cholera}
\subsection{Treatment Plan 1: Sanitation of water over time}
One of main ways control strategies for cholera is to treat the water directly (eg. with chlorine). This would essentially have the effect of increasing the rate of bacteria removal from water (defined as $\sigma$ in the base single patch model). This can be modeled by incorporating a new term $\rho$ in the base model:
\begin{linenomath}
\begin{align*}
	\frac{dS}{dt}&= \mu N - \mu S - b_i SI - b_w S W  \\
	\frac{dS}{dt}&= b_i S I + b_w S W - I (\gamma + \mu + \alpha) \\
	\frac{dR}{dt}&= \gamma I - \mu R \\
	\frac{dW}{dt}&= \xi I  - \sigma W - \rho W\\
\end{align*}
\end{linenomath}

This new term $\rho$ can be either a constant which is implemented right at the start, or can be a function of time, or can be implemented at a certain threshold depending on the bacterial concentration (W) or the proportion of infecteds (I). For example:
$$\rho (I)= \begin{cases}
	\rho & I \geq 0.1 \\
	0 & 0 \leq I \leq 0.1 \\
	\end{cases}$$\\
This represents the sanitation rate of $\rho$, implemented at certain threshold of infected (in this case the threshold is based on $I=0.1$ but can be based on W (i.e. testing water levels for cholera).\\


\subsection{Treatment Plan 2: Vaccinations on Base Model}

\begin{linenomath}
\begin{align*}
	\frac{dS}{dt}&= \mu N - \mu S - \beta_i SI - \beta_w S W - \nu S \\
	\frac{dI}{dt}&= \beta_i S I + \beta_w S W - I (\gamma + \mu + \alpha) \\
	\frac{dR}{dt}&= \gamma I - \mu R + \nu S\\
	\frac{dW}{dt}&= \xi I  - \sigma W\\
\end{align*}
\end{linenomath}

\begin{itemize}
	\item $\nu=$ is vaccination rate on S class
\end{itemize}

\subsection{Treatment Plan 3: Antibiotics on Base Model}

\begin{linenomath}
\begin{align*}
	\frac{dS}{dt}&= \mu N - \mu S - \beta_i SI - \beta_w S W \\
	\frac{dI}{dt}&= \beta_i S I + \beta_w S W - I (\gamma +\eta + \mu + \alpha ) \\
	\frac{dR}{dt}&= (\gamma +\eta)I - \mu R \\
	\frac{dW}{dt}&= \xi I  - \sigma W\\
\end{align*}
\end{linenomath}
\begin{itemize}
	\item $\eta=$ is antibiotic rate on I class
\end{itemize}

\section{Comparing Treatment Strategies For Cholera}
\subsection{Numerical Simulations and Phase Portraits For The Single Patch Model}

%\subsection{Numerical Simulations and Phase Portraits of Base and Treatment Models}

The following are numerical simulations and phase portraits for the base model with vital dynamics, and the three treatment models.


\begin{knitrout}
\definecolor{shadecolor}{rgb}{0.969, 0.969, 0.969}\color{fgcolor}\begin{figure}

{\centering \includegraphics[width=6cm]{figure/num_sim-1} 
\includegraphics[width=6cm]{figure/num_sim-2} 
\includegraphics[width=6cm]{figure/num_sim-3} 
\includegraphics[width=6cm]{figure/num_sim-4} 

}

\caption{\label{fig:num.sim} Plot of the SIRW model for a single patch and various treatment models. Parameters are $\mu=0.15\ \beta_i=0.06\ \gamma=0.14\ \sigma=0.07\ \beta_w=0.15\ \xi=10\ \rho=0.005\ \nu=0.01\ \eta=1e-04\ \alpha=0$. The initial conditions for the model were $S_0=0.995\ I_0=0.005\ R_0=0$}\label{fig:num.sim}
\end{figure}


\end{knitrout}

\begin{knitrout}
\definecolor{shadecolor}{rgb}{0.969, 0.969, 0.969}\color{fgcolor}\begin{figure}

{\centering \includegraphics[width=6cm]{figure/phase_portraits-1} 
\includegraphics[width=6cm]{figure/phase_portraits-2} 
\includegraphics[width=6cm]{figure/phase_portraits-3} 
\includegraphics[width=6cm]{figure/phase_portraits-4} 

}

\caption{\label{fig:phase.portraits} Phase portraits for a single patch and various treatment models. Parameters are $\mu=0.15\ \beta_i=0.06\ \gamma=0.14\ \sigma=0.07\ \beta_w=0.15\ \xi=10\ \rho=0.005\ \nu=0.01\ \eta=1e-04\ \alpha=0$. The initial conditions for the model were $S_0=0.995\ I_0=0.005\ R_0=0$}\label{fig:phase.portraits}
\end{figure}


\end{knitrout}

\FloatBarrier

From the phase portraits in figure \ref{fig:phase.portraits}, and the time series in \ref{fig:num.sim} it is clear that all treatments appear to have an effect on transmission dynamics, causing the infection to peak at a much higher level of incidence much quicker than without the treatment.
However the infectives also appear to recover comparably quickly as well, as seen by the sharp drop off in $I$.
All treatments appear to show very similar effects for the single patch model, but how work in the spatial model may differ, espescially singe the treatments can be applied heterogenously across the patches, in proportion to the initial infectives of each patch.

We hope to compare the effectiveness of these methods by comparing the final size, which can be computed from ${\mathcal R_0}$ in $R$ using Lambert's $W$ function, as noted in the supplementary material of \cite{link20}.
\begin{linenomath}
\begin{equation}
    Z({\mathcal R_0}) = 1+\frac{1}{{\mathcal R_0}}W(-{\mathcal R_0}e^{{\mathcal R_0}})
\end{equation}
\end{linenomath}
If similar final sizes are estimated for multiple treatment strategies, then relative costs of the treatments may be compared to decide between them.
Cost is not necessairily only monetary, as risk is associted with overuse of antibiotics, water sanitation requires maintenance, all strategies require work from various healthcare or engineering professions.
Further work and research needs to be done to define a more formal, specific cost comparison scheme in this case.
%Further peak prevelance can estimated from the initial conditions $S_0,I_0$ and ${\mathcal R_0}$ from following expression


%End Note

\bigskip\vfill
\centerline{\bf--- END OF PROJECT---}
\bigskip
Compile time for this document:
\today\ @ \thistime\\
CPU time to generate this document: 8.75S seconds.
\printbibliography
\end{document}
