%Preamble
\documentclass[12pt]{article}\usepackage[]{graphicx}\usepackage[]{color}
%% maxwidth is the original width if it is less than linewidth
%% otherwise use linewidth (to make sure the graphics do not exceed the margin)
\makeatletter
\def\maxwidth{ %
  \ifdim\Gin@nat@width>\linewidth
    \linewidth
  \else
    \Gin@nat@width
  \fi
}
\makeatother

\definecolor{fgcolor}{rgb}{0.345, 0.345, 0.345}
\newcommand{\hlnum}[1]{\textcolor[rgb]{0.686,0.059,0.569}{#1}}%
\newcommand{\hlstr}[1]{\textcolor[rgb]{0.192,0.494,0.8}{#1}}%
\newcommand{\hlcom}[1]{\textcolor[rgb]{0.678,0.584,0.686}{\textit{#1}}}%
\newcommand{\hlopt}[1]{\textcolor[rgb]{0,0,0}{#1}}%
\newcommand{\hlstd}[1]{\textcolor[rgb]{0.345,0.345,0.345}{#1}}%
\newcommand{\hlkwa}[1]{\textcolor[rgb]{0.161,0.373,0.58}{\textbf{#1}}}%
\newcommand{\hlkwb}[1]{\textcolor[rgb]{0.69,0.353,0.396}{#1}}%
\newcommand{\hlkwc}[1]{\textcolor[rgb]{0.333,0.667,0.333}{#1}}%
\newcommand{\hlkwd}[1]{\textcolor[rgb]{0.737,0.353,0.396}{\textbf{#1}}}%
\let\hlipl\hlkwb

\usepackage{framed}
\makeatletter
\newenvironment{kframe}{%
 \def\at@end@of@kframe{}%
 \ifinner\ifhmode%
  \def\at@end@of@kframe{\end{minipage}}%
  \begin{minipage}{\columnwidth}%
 \fi\fi%
 \def\FrameCommand##1{\hskip\@totalleftmargin \hskip-\fboxsep
 \colorbox{shadecolor}{##1}\hskip-\fboxsep
     % There is no \\@totalrightmargin, so:
     \hskip-\linewidth \hskip-\@totalleftmargin \hskip\columnwidth}%
 \MakeFramed {\advance\hsize-\width
   \@totalleftmargin\z@ \linewidth\hsize
   \@setminipage}}%
 {\par\unskip\endMakeFramed%
 \at@end@of@kframe}
\makeatother

\definecolor{shadecolor}{rgb}{.97, .97, .97}
\definecolor{messagecolor}{rgb}{0, 0, 0}
\definecolor{warningcolor}{rgb}{1, 0, 1}
\definecolor{errorcolor}{rgb}{1, 0, 0}
\newenvironment{knitrout}{}{} % an empty environment to be redefined in TeX

\usepackage{alltt}
\usepackage{scrtime} % for \thistime (this package MUST be listed first!)
\usepackage[margin=1in]{geometry}
\usepackage{graphics,graphicx}
\usepackage{placeins}
\usepackage[displaymath, mathlines]{lineno}
\usepackage{color}
\definecolor{aqua}{RGB}{0, 128, 225}
\usepackage[colorlinks=true,citecolor=aqua,linkcolor=aqua,urlcolor=aqua]{hyperref}
%For Math stuff
\usepackage{amsmath} % essential for cases environment
\usepackage{amsthm} % for theorems and proofs
\usepackage{amsfonts} % mathbb
%Referenes
\usepackage[natbib=true,
            style=nature,
            citestyle=authoryear,
            backend=biber,
            useprefix=true]{biblatex}
\addbibresource{./refs.bib}
%FANCY HEADER AND FOOTER STUFF
\usepackage{fancyhdr,lastpage}
\pagestyle{fancy}
\fancyhf{} % clear all header and footer parameters
%%%\lhead{Student Name: \theblank{4cm}}
%%%\chead{}
%%%\rhead{Student Number: \theblank{3cm}}
%%%\lfoot{\small\bfseries\ifnum\thepage<\pageref{LastPage}{CONTINUED\\on next page}\else{LAST PAGE}\fi}
\lfoot{}
\cfoot{{\small\bfseries Page \thepage\ of \pageref{LastPage}}}
\rfoot{}
\renewcommand\headrulewidth{0pt} % Removes funny header line
\IfFileExists{upquote.sty}{\usepackage{upquote}}{}
\begin{document}
%Title
\title{Examining Control Strategies for Cholera Incorporating Spatial Dynamics}
\author{
\underline{\emph{Group Name}}: \texttt{{\color{blue}The Plague Doctors}}\\\\
\underline{\emph{Group Members}}:\\
         Sid Reed\ :\ {\color{blue}reeds4@mcmaster.ca}\\
         Daniel Segura\ :\ {\color{blue}segurad@mcmaster.ca}\\
         Jessa Mallare\ :\ {\color{blue}mallarej@mcmaster.ca}\\
         Aref Jadda\ :\ {\color{blue}hossesa@mcmaster.ca}\\
}
\date{\today\ @ \thistime}
\maketitle
%due date
\bigskip
\noindent
This assignment is {\bfseries\color{red} due in class} on \textcolor{red}{\bf Wednesday March 27 2019 at 10:30am}.
\bigskip

\linenumbers

\begin{abstract}
We solve everything because we're really smart
\end{abstract}

\clearpage
\tableofcontents
\clearpage

%Sections
\section{Introduction}

It's time for a theory of everything.  Since we're all really smart, we've created one.

\section{Single Patch Models}
\subsection{Single Patch SIR Model With A Water Compartment}

%\section{Single Patch Model Of Cholera}
%	\textbf{Base Single Patch Model (with vital dynamics)}
\begin{linenomath}
	\begin{align*}
		\frac{dS}{dt}&= \mu N - \mu S - \beta_i SI - \beta_w S W  \\
		\frac{dI}{dt}&= \beta_i S I + \beta_w S W - I (\gamma + \mu + \alpha) \\
		\frac{dR}{dt}&= \gamma I - \mu R \\
		\frac{dW}{dt}&= \xi I  - \sigma W
	\end{align*}
\end{linenomath}

	\begin{itemize}
		\item$\mu=$ natural death rate
		\item$\beta_i=$ transmission rate between S and I class
		\item$\beta_w=$ transmission rate between I and W class
		\item$\gamma=$ recovery rate (I to R class)
		\item$\alpha=$ death rate from cholera
		\item$\xi=$ Shedding rate of cholera from I to W class
		\item$\sigma=$	Removal rate of cholera from W class (depends on what we define as our water source)
	\end{itemize}

\begin{itemize}
	\item This model assumes that you start off with low intensity symptoms (lower rate of shedding) and the symptoms reach a high intensity with a greater rate of shedding.
	\item$\alpha_i=$ death rate by cholera in low or high intensity
	\item$\delta =$ rate at which symptoms increase in severity
\end{itemize}


\begin{knitrout}
\definecolor{shadecolor}{rgb}{0.969, 0.969, 0.969}\color{fgcolor}\begin{figure}
\includegraphics[width=\maxwidth]{figure/_singlePatch-1} \caption{\label{fig:singlepatch} Plot of the SIRW model for a single patch. Parameters are $\mu=0.15\ \beta_i=0.06\ \gamma=0.14\ \sigma=0.07\ \beta_w=0.15\ \alpha=0$.Further the initial conditions for the model were $S_0=0.92\ I_0=0.08\ R_0=0$}\label{fig:<singlePatch}
\end{figure}


\end{knitrout}
\FloatBarrier
\subsection{Equilibrium and {$\mathcal R_0$} Of The Single Patch Model}

The basic reproductive number ${\mathcal R_0}$ is defined as the number of secondary infections as a result of a single infective during a time step.
${\mathcal R_0}$ can be computed as the spectral radius (i.e. the eigenvalue with the largest absolute value) of the next generation matrix at the disease free equilibrium.
The next generation matrix $FV^{−1}$, where the entry $F_{ij}$ of the matrix $F$ is the rate at which infected individuals in compartment $j$ produce new infections in compartment $i$, and the entry of $V_{ij}$ of the matrix $V$ is the mean time spent in compartment $j$ after moving into $j$ from compartment $k$.
For our model, we have
\begin{linenomath}
\begin{align*}
		F&=\begin{pmatrix}
			\beta_i & \beta_w\\
			0 & 0
			\end{pmatrix}\\
		V&=\begin{pmatrix}
			\frac{1}{\gamma+\mu+\alpha} & 0\\
			\frac{1}{\gamma+\mu+\alpha} &\frac{1}{\theta}
			\end{pmatrix}
\end{align*}
\end{linenomath}
The basic reproductive number is computed as the spectral radius of $FV^{-1}$ as seen in \cite{link9}, which is
\begin{linenomath}
\begin{align*}
    {\mathcal R_0} &= \rho(FV^{-1}\\
		           &=\frac{\beta_i+\beta_w}{\gamma+\mu}
\end{align*}
\end{linenomath}
This singla patch model has a disease-free equillibrium at (S,I,R)=(1,0,0) when ${\mathcal R_0}<1$.
It also has an endemic-equillirbium when ${\mathcal R_0}>1$

\subsection{Single Patch With Low And High Shedding Compartments}

%\textbf{Single Patch Model: Severity of Shedding dependent on Intensity of Symptoms (Low and High)}
\begin{linenomath}
\begin{align*}
	\frac{dS}{dt}&= \mu N - \mu S - \beta_L S I_L - \beta_H S I_H - \beta_w S W  \\
	\frac{dI_L}{dt}&= \beta_i S( I_L + I_H) + \beta_w S W - I_L (\mu + \delta + \alpha_L) \\
	\frac{dI_H}{dt}&= \delta I_L - I_H (\gamma + \mu + \alpha_H) \\
	\frac{dR}{dt}&= \gamma I_H - \mu R \\
	\frac{dW}{dt}&= \xi_L I_L + \xi_H I_H  - \sigma W\\
	\end{align*}
\end{linenomath}

\begin{itemize}
	\item This model assumes that you start off with low intensity symptoms (lower rate of shedding) and the symptoms reach a high intensity with a greater rate of shedding.
	\item$\alpha_i=$ death rate by cholera in low or high intensity
	\item$\delta =$ rate at which symptoms increase in severity
\end{itemize}

\section{Multi Patch Model}

%\section{Multi-Patch Models Of Cholera}
\begin{linenomath}
\begin{align*}
    \frac{dS}{dt}&= \mu N - \mu S - \beta_i SI - \beta_w S W  \\
    \frac{dI}{dt}&= \beta_i S I + \beta_w S W - I (\gamma + \mu + \alpha) \\
    \frac{dR}{dt}&= \gamma I - \mu R \\
    \frac{dW}{dt}&= \xi I  - \sigma W
\end{align*}
\end{linenomath}

\begin{itemize}
    \item$\mu=$ natural death rate
    \item$\beta_i=$ transmission rate between S and I class
    \item$\beta_w=$ transmission rate between I and W class
    \item$\gamma=$ recovery rate (I to R class)
    \item$\alpha=$ death rate from cholera
    \item$\xi=$ Shedding rate of cholera from I to W class
    \item$\sigma=$	Removal rate of cholera from W class (depends on what we define as our water source)
\end{itemize}


\begin{knitrout}
\definecolor{shadecolor}{rgb}{0.969, 0.969, 0.969}\color{fgcolor}\begin{figure}
\includegraphics[width=\maxwidth]{figure/_multiPatch-1} \caption{\label{fig:multipatch} Plot of the SIRW model for all patches in a multi patch model. Parameters are $\mu=0.15\ \beta_i=0.06\ \gamma=0.14\ \sigma=0.07\ \beta_w=0.15\ \alpha=0$. The initial conditions for the model were $S_0=0.97\ I_0=0.03\ R_0=0$. The influence of neighbouring patches is 0.15.}\label{fig:<multiPatch}
\end{figure}


\end{knitrout}
\FloatBarrier
\section{Treatment Strategies For Cholera}

%\section{Possible Treatment Strategies for Cholera}
\subsection{Treatment Plan 1: Sanitation of water over time}

\begin{linenomath}
\begin{align*}
	\frac{dS}{dt}&= \mu N - \mu S - \beta_i SI - \beta_w S W  \\
	\frac{dI}{dt}&= \beta_i S I + \beta_w S W - I (\gamma + \mu + \alpha) \\
	\frac{dR}{dt}&= \gamma I - \mu R \\
	\frac{dW}{dt}&= \xi I  - \sigma W - \rho (I) W\\
\end{align*}
\end{linenomath}

\begin{itemize}
	\item $\rho (I)= \begin{cases}
			 			\lambda & I \geq 0.1 \\
			 			0 & 0 \leq I \leq 0.1 \\
			 			\end{cases}$\\
	Represents the sanitation (increased removal of cholera) rate of $\lambda$, implemented at certain threshold of infected (in this case the threshold is based on I but can be based on W (i.e. testing water levels for cholera) %Can Change this part if needed
\end{itemize}


\subsection{Treatment Plan 2: Vaccinations on Base Model}

\begin{linenomath}
\begin{align*}
	\frac{dS}{dt}&= \mu N - \mu S - \beta_i SI - \beta_w S W - \nu S \\
	\frac{dI}{dt}&= \beta_i S I + \beta_w S W - I (\gamma + \mu + \alpha) \\
	\frac{dR}{dt}&= \gamma I - \mu R + \nu S\\
	\frac{dW}{dt}&= \xi I  - \sigma W\\
\end{align*}
\end{linenomath}

\begin{itemize}
	\item $\nu=$ is vaccination rate on S class
\end{itemize}

\subsection{Treatment Plan 3: Antibiotics on Base Model}

\begin{linenomath}
\begin{align*}
	\frac{dS}{dt}&= \mu N - \mu S - \beta_i SI - \beta_w S W \\
	\frac{dI}{dt}&= \beta_i S I + \beta_w S W - I (\gamma +\eta + \mu + \alpha ) \\
	\frac{dR}{dt}&= (\gamma +\eta)I - \mu R \\
	\frac{dW}{dt}&= \xi I  - \sigma W\\
\end{align*}
\end{linenomath}
\begin{itemize}
	\item $\eta=$ is antibiotic rate on I class
\end{itemize}

\section{Comparing Treatment Strategies For Cholera}


%End Note

\bigskip\vfill
\centerline{\bf--- END OF PROJECT---}
\bigskip
Compile time for this document:
\today\ @ \thistime\\
CPU time to generate this document: 1.148S seconds.
\printbibliography
\end{document}
