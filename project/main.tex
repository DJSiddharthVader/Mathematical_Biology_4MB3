%Preamble
\documentclass[12pt]{article}\usepackage[]{graphicx}\usepackage[]{color}
%% maxwidth is the original width if it is less than linewidth
%% otherwise use linewidth (to make sure the graphics do not exceed the margin)
\makeatletter
\def\maxwidth{ %
  \ifdim\Gin@nat@width>\linewidth
    \linewidth
  \else
    \Gin@nat@width
  \fi
}
\makeatother

\definecolor{fgcolor}{rgb}{0.345, 0.345, 0.345}
\newcommand{\hlnum}[1]{\textcolor[rgb]{0.686,0.059,0.569}{#1}}%
\newcommand{\hlstr}[1]{\textcolor[rgb]{0.192,0.494,0.8}{#1}}%
\newcommand{\hlcom}[1]{\textcolor[rgb]{0.678,0.584,0.686}{\textit{#1}}}%
\newcommand{\hlopt}[1]{\textcolor[rgb]{0,0,0}{#1}}%
\newcommand{\hlstd}[1]{\textcolor[rgb]{0.345,0.345,0.345}{#1}}%
\newcommand{\hlkwa}[1]{\textcolor[rgb]{0.161,0.373,0.58}{\textbf{#1}}}%
\newcommand{\hlkwb}[1]{\textcolor[rgb]{0.69,0.353,0.396}{#1}}%
\newcommand{\hlkwc}[1]{\textcolor[rgb]{0.333,0.667,0.333}{#1}}%
\newcommand{\hlkwd}[1]{\textcolor[rgb]{0.737,0.353,0.396}{\textbf{#1}}}%
\let\hlipl\hlkwb

\usepackage{framed}
\makeatletter
\newenvironment{kframe}{%
 \def\at@end@of@kframe{}%
 \ifinner\ifhmode%
  \def\at@end@of@kframe{\end{minipage}}%
  \begin{minipage}{\columnwidth}%
 \fi\fi%
 \def\FrameCommand##1{\hskip\@totalleftmargin \hskip-\fboxsep
 \colorbox{shadecolor}{##1}\hskip-\fboxsep
     % There is no \\@totalrightmargin, so:
     \hskip-\linewidth \hskip-\@totalleftmargin \hskip\columnwidth}%
 \MakeFramed {\advance\hsize-\width
   \@totalleftmargin\z@ \linewidth\hsize
   \@setminipage}}%
 {\par\unskip\endMakeFramed%
 \at@end@of@kframe}
\makeatother

\definecolor{shadecolor}{rgb}{.97, .97, .97}
\definecolor{messagecolor}{rgb}{0, 0, 0}
\definecolor{warningcolor}{rgb}{1, 0, 1}
\definecolor{errorcolor}{rgb}{1, 0, 0}
\newenvironment{knitrout}{}{} % an empty environment to be redefined in TeX

\usepackage{alltt}
\usepackage{scrtime} % for \thistime (this package MUST be listed first!)
\usepackage[margin=1in]{geometry}
\usepackage{graphics,graphicx}
\usepackage{lineno}
\usepackage{color}
\definecolor{aqua}{RGB}{0, 128, 225}
\usepackage[colorlinks=true,citecolor=aqua,linkcolor=aqua,urlcolor=aqua]{hyperref}
%For Math stuff
\usepackage{amsmath} % essential for cases environment
\usepackage{amsthm} % for theorems and proofs
\usepackage{amsfonts} % mathbb
%Referenes
\usepackage[natbib=true,
            style=nature,
            backend=biber,
            useprefix=true]{biblatex}
\addbibresource{./refs.bib}
%FANCY HEADER AND FOOTER STUFF
\usepackage{fancyhdr,lastpage}
\pagestyle{fancy}
\fancyhf{} % clear all header and footer parameters
%%%\lhead{Student Name: \theblank{4cm}}
%%%\chead{}
%%%\rhead{Student Number: \theblank{3cm}}
%%%\lfoot{\small\bfseries\ifnum\thepage<\pageref{LastPage}{CONTINUED\\on next page}\else{LAST PAGE}\fi}
\lfoot{}
\cfoot{{\small\bfseries Page \thepage\ of \pageref{LastPage}}}
\rfoot{}
\renewcommand\headrulewidth{0pt} % Removes funny header line
\IfFileExists{upquote.sty}{\usepackage{upquote}}{}
\begin{document}
%Title
\title{Examining Control Strategies for Cholera Incorporating Spatial Dynamics}
\author{
\underline{\emph{Group Name}}: \texttt{{\color{blue}The Plague Doctors}}\\\\
\underline{\emph{Group Members}}: {\color{blue}Sid Reed, Daniel Segura, Jessa Mallare, Aref Jadda}}
\date{\today\ @ \thistime}
\maketitle
%due date
\bigskip
\noindent
This assignment is {\bfseries\color{red} due in class} on \textcolor{red}{\bf Wednesday March 27 2019 at 10:30am}.
\bigskip

\linenumbers

\begin{abstract}
We solve everything because we're really smart
\end{abstract}

\clearpage
\tableofcontents
\clearpage
%Sections child

\section{Background}
It's time for a theory of everything.  Since we're all really smart, we've created one.


\section{Model Description And Biological Processes}
\begin{align*}
    dS_i&=\mu N_i-S_1(\beta_n^i+\kappa W_i+\mu)\\
    dI_i&=S_i(\beta_n^i+\kappa W_i)-I_i(\gamma+\mu+\alpha)\\
    dR_i&=\gamma I_i-\mu R_i\\
    dW_i&=\beta_v I_i+\sum_{1}^{j}{\Big(1-\frac{W_j}{dist_{i,j}}\Big)} -\sigma W_i\\
    \beta_n^i&=\sum_{j}^{n}{\beta_t\Big(1-\frac{dist(i,j)}{max dist}\Big)I_j}+\beta_iI_i
\end{align*}
\begin{itemize}
  \item $\mu=$ natural birth/death rate
  \item $\beta_n^i=$infectivity of all neighbours of i on i
  \item$\gamma=$rate of recovery from disease
  \item$\beta_v=$rate infectious people transmit cholera to water
  \item$\sigma=$ rate of water sanitation/cholera death
  \item$\alpha=$ death rate from cholera
  \item$\beta_t=$transmisison rate within a patch
  \item $\kappa=$ rate at which 1 unit of chlera infects people
\end{itemize}

\section{Multipatch Models Of Cholera}
There is one equation for our theory:
\begin{linenomath*}
\begin{equation}\label{E:U}
U = 0 \,.
\end{equation}
\end{linenomath*}
We leave it as an exercise for the reader to define $U$.
We exploit Euler's formula,
\begin{linenomath*}
\begin{equation}\label{E:Euler}
e^{i\pi} + 1 = 0 \,.
\end{equation}
\end{linenomath*}
%Setting seed globally
\begin{knitrout}
\definecolor{shadecolor}{rgb}{0.969, 0.969, 0.969}\color{fgcolor}\begin{kframe}
\begin{alltt}
\hlcom{#Setting the seed for the entire document, for reproducible stochastic simulations}
\hlstd{seed} \hlkwb{<-} \hlnum{9}
\hlkwd{set.seed}\hlstd{(seed)}
\end{alltt}
\end{kframe}
\end{knitrout}

\section{Comparing Containment Strategies}
test \cite{link1}
test \cite{link2}
test \cite{link3}
test \cite{link4}
test \cite{link5}
test \cite{link6}
test \cite{link7}
test \cite{link8}
test \cite{link9}
test \cite{link10}
test \cite{link11}
test \cite{link12}
test \cite{link13}
test \cite{link14}
test \cite{link15}
test \cite{link16}


\section{Discussion}
This is really important stuff.

%Sections sexpr
%knit_child('./sections/background.Rnw')
%knit_child('./sections/model_description.Rnw')
%knit_child('./sections/multipatch.Rnw')
%knit_child('./sections/results.Rnw')
%knit_child('./sections/discussion.Rnw')
%End Note
\bigskip\vfill
\centerline{\bf--- END OF PROJECT---}
\bigskip
Compile time for this document:
\today\ @ \thistime
\printbibliography
\end{document}
