\newpage
\section{Executive Summary}
\par Cholera is a water-borune disease  that affects three to five million people and causes 100,000 - 120,000 deaths every year and is a major health concern in areas that have poor water sanitation. In case of an outbreak the there are three treatment strategies that are usually implemented, sanitation, vaccination, and antibiotic treatment. In order to investigate which treatment strategies would work best in the event of a cholera outbreak, the disease dynamics of cholera were discussed and used to form a SIRW model in this paper. This model was used as a base onto which the three main treatment strategies for cholera outbreaks could also be modelled and examined. When analyzed, it was found that the best treatment strategy inthe case of a cholera outbreak using the SIRW model appears to be sanitation of infected water sources. 
\par A layer of complexity that was incorporated into the model was the spatial componenet to cholera disease dynamics. A spatial model better reflects the spread of the disease between individuals and the infection of water sources. We also incorporated a disease severity component into the model, where only those that were infected, but not severely ill could travel and spread the infection. This was done to better mimic an outbreak scenario, as individuals that are severely ill do not often travel. This spatial model used to analyze the effect of the treatment strategies once more, and in this case it was found that sanitation of infected water sources seems most effective, but combination therapies should also be investigated in the future. Furthermore, the dynamics of this proposed model could be adjusted to model the spread of diseases that are similar in nature to cholera, such as other water-borne diseases.
