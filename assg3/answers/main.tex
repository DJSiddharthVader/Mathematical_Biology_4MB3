\documentclass[12pt]{article}\usepackage[]{graphicx}\usepackage[]{color}
%% maxwidth is the original width if it is less than linewidth
%% otherwise use linewidth (to make sure the graphics do not exceed the margin)
\makeatletter
\def\maxwidth{ %
  \ifdim\Gin@nat@width>\linewidth
    \linewidth
  \else
    \Gin@nat@width
  \fi
}
\makeatother

\definecolor{fgcolor}{rgb}{0.345, 0.345, 0.345}
\newcommand{\hlnum}[1]{\textcolor[rgb]{0.686,0.059,0.569}{#1}}%
\newcommand{\hlstr}[1]{\textcolor[rgb]{0.192,0.494,0.8}{#1}}%
\newcommand{\hlcom}[1]{\textcolor[rgb]{0.678,0.584,0.686}{\textit{#1}}}%
\newcommand{\hlopt}[1]{\textcolor[rgb]{0,0,0}{#1}}%
\newcommand{\hlstd}[1]{\textcolor[rgb]{0.345,0.345,0.345}{#1}}%
\newcommand{\hlkwa}[1]{\textcolor[rgb]{0.161,0.373,0.58}{\textbf{#1}}}%
\newcommand{\hlkwb}[1]{\textcolor[rgb]{0.69,0.353,0.396}{#1}}%
\newcommand{\hlkwc}[1]{\textcolor[rgb]{0.333,0.667,0.333}{#1}}%
\newcommand{\hlkwd}[1]{\textcolor[rgb]{0.737,0.353,0.396}{\textbf{#1}}}%
\let\hlipl\hlkwb

\usepackage{framed}
\makeatletter
\newenvironment{kframe}{%
 \def\at@end@of@kframe{}%
 \ifinner\ifhmode%
  \def\at@end@of@kframe{\end{minipage}}%
  \begin{minipage}{\columnwidth}%
 \fi\fi%
 \def\FrameCommand##1{\hskip\@totalleftmargin \hskip-\fboxsep
 \colorbox{shadecolor}{##1}\hskip-\fboxsep
     % There is no \\@totalrightmargin, so:
     \hskip-\linewidth \hskip-\@totalleftmargin \hskip\columnwidth}%
 \MakeFramed {\advance\hsize-\width
   \@totalleftmargin\z@ \linewidth\hsize
   \@setminipage}}%
 {\par\unskip\endMakeFramed%
 \at@end@of@kframe}
\makeatother

\definecolor{shadecolor}{rgb}{.97, .97, .97}
\definecolor{messagecolor}{rgb}{0, 0, 0}
\definecolor{warningcolor}{rgb}{1, 0, 1}
\definecolor{errorcolor}{rgb}{1, 0, 0}
\newenvironment{knitrout}{}{} % an empty environment to be redefined in TeX

\usepackage{alltt}
\input{preamble.tex}
\input{4mba3q.tex}
\usepackage{amsthm}
%\theoremstyle{definition}
\newtheorem{definition}{Definition}

%% FANCY HEADER AND FOOTER STUFF %%
\usepackage{fancyhdr,lastpage}
\pagestyle{fancy}
\fancyhf{} % clear all header and footer parameters
%%%\lhead{Student Name: \theblank{4cm}}
%%%\chead{}
%%%\rhead{Student Number: \theblank{3cm}}
%%%\lfoot{\small\bfseries\ifnum\thepage<\pageref{LastPage}{CONTINUED\\on next page}\else{LAST PAGE}\fi}
\lfoot{}
%Referenes
\usepackage[natbib=true,
            style=nature,
            backend=biber,
            useprefix=true]{biblatex}
\addbibresource{./refs.bib}

\cfoot{{\small\bfseries Page \thepage\ of \pageref{LastPage}}}
\rfoot{}
\renewcommand\headrulewidth{0pt} % Removes funny header line
\IfFileExists{upquote.sty}{\usepackage{upquote}}{}
\begin{document}
%HEADER
\begin{center}
{\bf Mathematics 4MB3/6MB3 Mathematical Biology\\
\smallskip
\url{http://www.math.mcmaster.ca/earn/4MB3}\\
\smallskip
2019 ASSIGNMENT 3}\\
\medskip
\underline{\emph{Group Name}}: \texttt{{\color{blue}The Plague Doctors}}\\
\medskip
\underline{\emph{Group Members}}: {\color{blue}Sid Reed, Daniel Segura, Jessa Mallare, Aref Jadda}
\end{center}
\bigskip
\noindent
This assignment is {\bfseries\color{red} due in class} on \textcolor{red}{\bf Monday 25 February 2019 at 9:30am}.
\bigskip

\section*{Analysis of the standard SIR model with vital dynamics}
\SIRintro

\begin{enumerate}[(a)]
\item %Size constancy
Since equations (1) represent all changes in the size of each population compartment, the net change in the total population should be the sum of the change in each compartment, i.e. the sum of all equations (1).
If the sum of all equations (1) is zero, $\frac{dS}{dt} + \frac{dI}{dt} + \frac{dR}{dt} = 0$, the the change in total population size must be zero and the total population size $N$ must be constant.
\begin{align}
   \frac{dS}{dt} + \frac{dI}{dt} + \frac{dR}{dt} &= \mu N -\frac{\beta}{N} SI - \mu S + \frac{\beta}{N} SI - \gamma I - \mu I + \gamma I - \mu R\\
   &= \mu N - \mu S - \mu I - \mu R (-\frac{\beta}{N} SI + \frac{\beta}{N} SI) (- \gamma I  + \gamma I)\\
   &= \mu (N - (S + I + R)) \label{qa1}
\end{align}
Since $N = S + I + R$, i.e. the sum of all population compartments is equal to the total population size, \ref{qa1} evaluates to 0.
Thus the sum of population changes in all population compartments is 0 and the total population size remains constant.\\\\
%Forward Invariance
\begin{definition}{Forward Invariant Set\\}
Given a dynamical system $\dot{x}=f(x)$, a solution $x(t,x_0)$ with initial condition $x_0$, a set $\Delta = \{x \in {\mathbb R}\ | \phi(x) = 0 \}$ for some positive definite function $\phi(x)$ is forward invariant if $x_0 \in \Delta \implies x(t,x_0) \in \Delta\ \forall\ t \geq 0$.
%$x_0 \in S \implies \varphi(t,x_0) \in S\ \forall\ t \in I(x_0)\ \textrm{and}\ t \geq 0$, where $I(x_0)$ is the maximum interval of existence for the solution and $\varphi(t,x_0)$ is a solution to the ODE with initial condition $x_0$ after time $t$.
\end{definition}
Since the population size has been shown to be constant and equal to $N$, the function $\phi(S,I,R) = N - (S+I+R)$ is always equal to zero, given any initial condition $(S,I,R)$ satisfying $0 \leq S,I,R \leq N$ and $S+I+R = N$.
%Bio States
\begin{definition}{Biologically Meaningful States\\}
Define the set $\Delta = \{(S,I,R)\ |\ 0 \leq S,I,R \ \textrm{and}\ \phi(S,I,R) = 0\}$ where $\phi(S,I,R) = N - (S+I+R)$, to be the set of biologically meaningful states for this model.
\end{definition}
Once the total population is equal to $N$, it will remain equal to $N$ in all subsequent time steps due to the population size constancy.
If all initial conditions are defined such that they satisfy $\phi(S,I,R) = 0$, then they can only evolve towards other states that satisfy $\phi(S,I,R) = 0$ due to the constant population size.
Thus if the set $\Delta$ is defined to include all initial conditions $x_0$ that have a total population equal to $N$ (i.e. all 3-tuples of positive integers $x_0 = (S,I,R)$ such that $\phi(S,I,R) = 0$), then they must necessarily include all possible time steps for solutions to the dynamical system with initial condition $x_0$, since the total population must remain constant over all time.

\item Set the following variables:
\begin{subequations}\label{pvars}
\begin{align}
    S_p &= \frac{S}{N}\\
    I_p &= \frac{I}{N}\\
    R_p &= \frac{R}{N}\\
    N_p &= \frac{N}{N} = 1
\end{align}
\end{subequations}
Then substituting equations \ref{pvars} into equations (1)
\begin{align}
  \frac{dS_p}{dt} &= \mu N_p -\frac{\beta}{N_p} S_p I_p - \mu S_p\nonumber\\
                  &= \mu 1 -\frac{\beta}{1} S_p I_p - \mu S_p\nonumber\\
                  &= \mu -\beta S_p I_p - \mu S_p\nonumber\\
                  &= \mu -\frac{\beta}{N} \frac{S I}{N} - \mu \frac{S}{N}\nonumber\\
                  &= \frac{1}{N}(\mu N -\frac{\beta}{N} SI - \mu S)\nonumber\\
                  &= \frac{1}{N} \frac{dS}{dt}\label{sp}\\
  \vspace{8pt}
  \frac{dI_p}{dt} &= \frac{\beta}{N_p} S_p I_p - \gamma I_p - \mu I_p\nonumber\\
                  &= \frac{\beta}{1} S_p I_p - \gamma I_p - \mu I_p\nonumber\\
                  &= \frac{\beta}{N} \frac{S I}{N} - \gamma \frac{I}{N} - \mu \frac{I}{N}\nonumber\\
                  &= \frac{1}{N} (\frac{\beta}{N} SI - \gamma I - \mu I)\nonumber\\
                  &= \frac{1}{N} \frac{dI}{dt}\label{ip}\\
  \vspace{8pt}
  \frac{dR_p}{dt} &= \gamma I_p - \mu R_p\nonumber\\
  \frac{dR_p}{dt} &= \gamma \frac{I}{N} - \mu \frac{R}{N}\nonumber\\
                  &= \frac{1}{N} (\gamma I - \mu R)\nonumber\\
                  &= \frac{1}{N} \frac{dR}{dt}\label{rp}
\end{align}
From equations \ref{sp},\ref{ip} and \ref{rp} it is clear that the proportional equations are equivalent to the original equations (1) scaled by a constant factor of $\frac{1}{N}$, and thus will retain the same dynamical behaviour.

\item %d/dtau
First we express $dt$ in terms of $\tau$
\begin{align}
    \tau &= t(\gamma + \mu)\nonumber\\
    d\tau &= dt(\gamma + \mu)\nonumber\\
    \frac{d\tau}{dt} &= (\gamma + \mu)\nonumber\\
    \frac{d\tau}{dt}\frac{d}{d\tau} &= \frac{d}{dt}\nonumber\\
    (\gamma+\mu)\frac{d}{d\tau} &= \frac{d}{dt}\nonumber\\
    \frac{d}{d\tau} &= \frac{1}{(\gamma+\mu)}\frac{d}{dt}\label{tau}
\end{align}
%isloate gamma,beta,mu
Next we isolate $\gamma,\beta,\mu$,
\begin{align}
    \eps &= \frac{\mu}{\gamma+\mu}\nonumber\\
    \eps(\gamma + \mu) &= \mu &\gamma+\mu = \frac{\mu}{\eps} \nonumber\\
    \eps \gamma + \eps \mu &= \mu\nonumber\\
    \eps \gamma &= \mu(1-\eps)\nonumber\\
    \gamma &= \frac{\mu}{\eps}(1-\eps)\nonumber\\
    \gamma &= (\gamma+\mu)(1-\eps)\label{gamma}\\
    \textrm{From (2b) it follows that}\ \beta &= (\gamma+\mu){\mathcal R_0}\label{beta}\\
    \textrm{From (2c) it follows that}\ \mu &= (\gamma+\mu)\eps\label{mu}
\end{align}
%dS/dt
Next, expressing $\frac{dS}{dt}$ in terms of $\tau$ using \ref{tau} and substituting equations \ref{gamma},\ref{beta} and \ref{mu} gives
\begin{align}
    \frac{dS}{d\tau} &= \frac{1}{\gamma+\mu}\frac{dS}{dt}\nonumber\\
     &= \frac{1}{\gamma+\mu}[\mu-\beta SI -\mu S]\nonumber\\
     &= \frac{1}{\gamma+\mu}[\mu(1-S)-\beta SI]\\
     &= \frac{1}{\gamma+\mu}[\eps(\gamma+\mu)(1-S)- (\gamma+\mu){\mathcal R_0}SI]\nonumber\\
     &= \frac{1}{\gamma+\mu}(\gamma+\mu)[\eps(1-S)- {\mathcal R_0}SI]\nonumber\\
     &= \eps(1-S)- {\mathcal R_0}SI
\end{align}
%dI/dt
Solving for $\frac{dI}{d\tau}$ using \ref{tau}, \ref{gamma}, \ref{beta} and \ref{mu} gives
\begin{align}
    \frac{dI}{d\tau} &= \frac{1}{\gamma+\mu}\frac{dI}{dt}\nonumber\\
     &= \frac{1}{\gamma+\mu}[\beta SI -\gamma I-\mu I]\nonumber\\
     &= \frac{1}{\gamma+\mu}[(\gamma+\mu){\mathcal R_0}SI -(\gamma+\mu)(1-\eps) I-(\gamma+\mu)\eps I]\nonumber\\
     &= \frac{1}{\gamma+\mu}(\gamma+\mu)[{\mathcal R_0}SI -(1-\eps) I-\eps I]\nonumber\\
     &= {\mathcal R_0}SI -(1-\eps) I-\eps I\nonumber\\
     &= {\mathcal R_0}SI -(1-2\eps) I
\end{align}
%dR/dt
Finally, solving for $\frac{dR}{d\tau}$ using \ref{tau}, \ref{gamma}, \ref{beta} and \ref{mu} gives
\begin{align*}
    \frac{dR}{d\tau} &= \frac{1}{\gamma+\mu}\frac{dR}{dt}\nonumber\\
     &= \frac{1}{\gamma+\mu}[\gamma I - \mu R]\nonumber\\
     &= \frac{1}{\gamma+\mu}[(\gamma+\mu)(1-\eps)I - (\gamma+\mu)\eps R]\nonumber\\
     &= \frac{1}{\gamma+\mu}(\gamma+\mu)[(1-\eps)I - \eps R]\nonumber\\
     &= (1-\eps)I - \eps R
\end{align*}
%Biological meanings
%1/(mu+gamma) is average time of infection
The biological meanings of $\tau, {\mathcal R_0}$ and $\eps$ are
\begin{itemize}
    \item $\tau$ is the average proportion of the population infected
    \item ${\mathcal R_0}$ is the number of secondary infections per infection
    \item $\eps$ is the mortality rate for the infected.
    %\item $\eps$ is the average period being infected until death or recovery
    %\item $\eps$ is the ratio of new people who can be infected (births) to people who can no longer be infected (recover or die), or the rate at which the succeptible class grows.
\end{itemize}
%justification for tau,epsilon,rnot

%disease epsilon
In the UK the morality rate from penumonia between 2001-2010 was estimated to be $0.0214\%$ of people\cite{pneu}.
Ebola Virus Diesase is estimated by the WHO to have an average case fatality rate of $50\%$\cite{ebola}.
The CDC estimates measles to have a mortality rate between $0.1\%$ to $0.2\%$ \cite{meas}.
%For ebola the mean infectious period ($\eps$) has been estimated to be $5.33\pm 4.03$ days from wet symptom onset until death.\cite{ebola}.
%For influenza the mean infectious period ($\eps$) has been estimated to be $1.0$ days until death.\cite{flu}. For other diseases $\eps$ appears to be on the order of days to weeks.

\item From our proportion equations, setting $\frac{dI}{dt} = 0$ we get:

\begin{align*}
	\beta SI - \gamma I - \mu I = 0 \\
	I (\beta S - \gamma - \mu ) = 0 \\
	\beta S =  \gamma + \mu \\
	\^{S} = \frac{1}{R_{0}}
\end{align*}

Adding the proportion equations of $\frac{dS}{dt} = 0$ and $\frac{dI}{dt} = 0$ we also get:

\begin{align*}
	\mu - \mu S - \gamma I - \mu I = 0 \\
	1 - S - I = \frac{\gamma}{\mu} I \\
	1 - S = \frac{\mu + \gamma}{\mu} I \\
	1 - S = \frac{1}{\epsilon} I \\
	I = \epsilon (1 - S) \\
	I = \epsilon (1 - \frac{1}{R_{0}})
\end{align*}

Therefore $(\^{S}, \^{I}) = (\frac{1}{R_{0}}, \epsilon - \frac{\epsilon}{{R_{0}}})$. Both equilibria are biologically relevant as long as $R_{0} \geq 1$ , since values of S and I outside of the range [0,1] are not meaningful as proportions.

\item Local asymptotic stability can be derived logically from the definition of $R_0$. When $R_0 < 1$, this means that the number of secondary infections per infection is less than 1. Thus, if the disease has just started, the number of infected will immediately go back to 0 or the disease free equilibrium.\\
On the other hand if $R_0 > 1$, the disease will spread once it starts. The number of infected (I) can never reach 1 either since $\epsilon$ (mortality rate) exists, so there is also an upper limit. Ergo, the equilibrium in between (0,1), or the endemic equilibrium, must be asymptotically stable when $R_0 > 1$.\\
A more mathematically detailed proof is done in parts f and g, proving global asymptotic stability under these conditions.
\item Let us start by using the Lyapunov function $L = I$.
$L = 0$ when $(S,I) = (1,0)$. Since $R_0 \leq 1$ we can conclude that $\beta \leq \gamma + \mu$ or $\beta - \gamma - \mu \leq 0$. Calculating $\dot{L}$ we have:

\begin{align*}
    \dot{L} &= \frac{dL}{dI} \frac{dI}{dt} + \frac{dL}{dS} \frac{dS}{dt} \\
          &= \beta SI - \gamma I - \mu I + \mu I - \beta S I^2 - \mu SI \\
          &= I (\beta S - \gamma - \mu S - \beta SI) \\
          &\leq I (\beta S - \gamma S - \mu S - \beta SI) \\
          &= I [S (\beta - \gamma - \mu) - \beta SI]\\
          &< 0 \forall\ S,I \in (0,1)
\end{align*}

$L(1,0) = 0$ and $\dot{L} < 0$ $\forall S,I \in (0,1)$. Hence, L is negative definite on the interval (0,1). L is therefore a strict Lyapunov function for the DFE on the entire biologically relevant space, which implies that the DFE is globally asymptotically stable.

\item The EE is $(\hat{S},\hat{I}) = (\frac{1}{{\mathcal R_0}},\eps(1-\frac{1}{{\mathcal R_0}})$.
Consider the Lyapunov function with ${\mathcal O} = \{(S,I) | 0\leq S,I \leq 1\ and\ S+I = 1\} $
\begin{align}
    L(S,I) &= S - \frac{1}{{\mathcal R_0}}log(S) + I - \eps(1-\frac{1}{{\mathcal R_0}})log(I)\nonumber\\
           &= S - \frac{1}{{\mathcal R_0}}log(S) + I - \eps log(I) -(\frac{\eps}{{\mathcal R_0}})log(I)\nonumber\\
           &= S - \frac{1}{{\mathcal R_0}}log(S) + I + \eps log(I) (\frac{1}{{\mathcal R_0}}-1)\label{lup}\\
    0 \leq S,I \leq 1 &\implies log(S), log(I) < 0 \implies \nonumber\\
    0   &< S - \frac{1}{{\mathcal R_0}}log(S) + I + \eps log(I) (\frac{1}{{\mathcal R_0}}-1)\label{pos}
\end{align}
Thus \ref{pos} shows that \ref{lup} is positive on $[0,1]$, the the function is postive definite on $[0,1]$.
\begin{align}
    L(\hat{S},\hat{I})  &= \hat{S} - \hat{S}log(\hat{S}) + \hat{I}\\
    &= \hat{S} - \frac{1}{{\mathcal R_0}}log(\frac{1}{{\mathcal R_0}}) + I - \eps(1-\frac{1}{{\mathcal R_0}})log(I)\label{lupeq}\\
\end{align}
From Lyapunov's theorem, a function $L(X)$ satisfies $L(X) > L(X_*) \forall\ X \in {\mathcal O} \setminus \{X_*\}$ where $X_* = (\hat{S},\hat{I})$, and using the same set ${\mathcal O}$ as defined earilier.
\begin{align}
    L(S,I) &> L(\hat{S},\hat{I})\nonumber\\
    S - \hat{S}log(S) +I - \hat{I}log(I) &> \hat{S} - \hat{S}log(\hat{S}) +\hat{I} - \hat{I}log(\hat{I})\nonumber
\end{align}
\begin{align}
    \frac{\partial L}{\partial S} &= 1 - \frac{\hat{S}}{S}\nonumber\\
    \frac{\partial L}{\partial S} &= 1 - \frac{\hat{S}}{S} = 0 \implies S = \hat{S}\nonumber\\
    \frac{\partial L}{\partial I} &= 1 - \frac{\hat{I}}{I}\nonumber\\
    \frac{\partial L}{\partial I} &= 1 - \frac{\hat{I}}{I} = 0 \implies I = \hat{I}\nonumber
\end{align}
The functions $log(S), log(I)$ increase monotonically with respect to $\hat{S},\hat{I}$. Both partial derivatives $\frac{\partial L}{\partial S}, \frac{\partial L}{\partial I}$ as well, the by \ref{mono1}, \ref{mono2} show that $(\hat{S},\hat{I}$ must be the global minimum of $L(S,I)$ on $[0,1]$.\\
Further
\begin{align}
    \frac{\partial^2 L}{\partial S^2} &= \hat{S}S^{-2} \textrm{at} \hat{S} = \frac{\hat{S}}{\hat{S}^2} = \frac{1}{\hat{S}}\nonumber\\
    \frac{\partial^2 L}{\partial I^2} &= \hat{I}I^{-2} \textrm{at} \hat{I} = \frac{\hat{I}}{\hat{I}^2} = \frac{1}{\hat{I}}\nonumber\\
\end{align}
are non-negative at the minimum point $(\hat{S},\hat{I})$, thus $L(S,I) > L(\hat{S},\hat{I}) \forall\ (S,I) \in {\mathcal O} \setminus \{(\hat{S},\hat{I})\}$.

\item %explanation
Since the EE is GAS, any oscillations will be damped and approaching the EE.
Oscillations will occur if the imaginary parts of the eigenvalues of the Jacobian, evaluated at the EE $(\hat{S},\hat{I}) = (\frac{1}{\mathcal R_0},\eps(1-\frac{1}{\mathcal R_0})$, are non-zero.
Since the Jacobian is a $2\times 2$ matrix, the eigenvalues are the roots of the equation $\lambda^2 - T\lambda + D = 0$ where T and D are the trace and determinant of the Jacobian respectively.
The roots can be found using the quadratic formula $\frac{-b \pm \sqrt{b^2-4ac}}{2a}$.
The imaginary part of the roots will only be non-zero if $\sqrt{b^2-4ac} < 0$, thus proving $\sqrt{b^2-4ac} < 0$ implies damped oscillations approaching the EE.\\
The Jacobian at the EE is
\begin{align}
J &=
\begin{bmatrix}
    \frac{\partial S}{\partial S} & \frac{\partial S}{\partial I}\\
    \frac{\partial I}{\partial S} & \frac{\partial I}{\partial I}
\end{bmatrix}\\
&=
\begin{bmatrix}
    -\beta I - \mu & -\beta S\\
    \beta I &\beta S - (\gamma+\mu)
\end{bmatrix}\\
J(\hat{S},\hat{I}) &= J(\frac{1}{\mathcal R_0},\eps(1-\frac{1}{\mathcal R_0})\\
&=
\begin{bmatrix}
    -\beta \eps(1-\frac{1}{\mathcal R_0}) - \mu & -\beta \frac{1}{\mathcal R_0}\\
    \beta \eps(1-\frac{1}{\mathcal R_0}) &\beta \frac{1}{\mathcal R_0} - (\gamma+\mu)
\end{bmatrix}\\
& \textrm{Next substitute $\beta, \gamma, \mu$ using equations \ref{gamma},\ref{beta},\ref{mu}}\nonumber\\
&=
\begin{bmatrix}
    -(\gamma+\mu){\mathcal R_0} \eps(1-\frac{1}{\mathcal R_0}) - \eps(\gamma+\mu) & -(\gamma+\mu){\mathcal R_0} \frac{1}{\mathcal R_0}\\
    (\gamma+\mu){\mathcal R_0} \eps(1-\frac{1}{\mathcal R_0}) &(\gamma+\mu){\mathcal R_0} \frac{1}{\mathcal R_0} - (\gamma+\mu)
\end{bmatrix}\\
&= (\gamma+\mu)
\begin{bmatrix}
    -\eps {\mathcal R_0}  & -1\\
    \eps({\mathcal R_0}-1) & 0
\end{bmatrix}
\end{align}
Next the trace $T$, determinant $D$, and the terms of the quadratic formula $a,b,c$ are
\begin{align}
    T &= -\eps {\mathcal R_0} + 0\\
    D &= (-\eps {\mathcal R_0})(0) - (-1)(\eps({\mathcal R_0}-1)\\
      &= \eps({\mathcal R_0}-1)\\
    0 &= \lambda^2 - T \lambda + D \implies a = 1\ \ b = -T\ \ c = D\\
    b &= \eps {\mathcal R_0}\\
    c &= \eps({\mathcal R_0}-1)\\
    \sqrt{b^2-4ac} &= \sqrt{\eps^2 {\mathcal R_0}^2 - 4(1)(\eps({\mathcal R_0}-1))}\\
                   &= \sqrt{\eps^2 {\mathcal R_0}^2 - 4(\eps({\mathcal R_0}+4\eps}\\
                   &= \sqrt{\eps(\eps{\mathcal R_0}^2 - 4{\mathcal R_0}+4)}\label{bsqr}
\end{align}
Now if we show that the when the inequality
\begin{equation}\label{eqee}
    \sqrt{\eps(\eps{\mathcal R_0}^2 - 4{\mathcal R_0}+4)} < 0
\end{equation}
holds it implies $\eps < \eps^* = \frac{4{\mathcal R_0}-1}{{\mathcal R_0}^2}$, then this proves that the approach to EE via damped oscillations occurs iff $\eps < \eps^*$
\begin{align}
    \sqrt{\eps(\eps{\mathcal R_0}^2 - 4{\mathcal R_0}+4)} &< 0\\
    \sqrt{\frac{4{\mathcal R_0}-1}{{\mathcal R_0}^2}(\frac{4({\mathcal R_0}-1)}{{\mathcal R_0}^2}{\mathcal R_0}^2 - 4{\mathcal R_0}+4)} &< 0\\
    \sqrt{\frac{4{\mathcal R_0}-1}{{\mathcal R_0}^2}(4({\mathcal R_0}-1) - 4{\mathcal R_0}+4)} &< 0\\
    \sqrt{\frac{4{\mathcal R_0}-1}{{\mathcal R_0}^2}(4{\mathcal R_0}-4 - 4{\mathcal R_0}+4)} &< 0\\
    \sqrt{\frac{4{\mathcal R_0}-1}{{\mathcal R_0}^2}(0)} &< 0\\
    0 &< 0 \textrm{which is false}
\end{align}
Since \ref{eeqe} is 0 when $\eps = \eps^*$, increasing $\eps$ will cause $\sqrt{\eps(\eps{\mathcal R_0}^2 - 4{\mathcal R_0}+4)}$ to become positive and decreasing $\eps$ will cause $\sqrt{\eps(\eps{\mathcal R_0}^2 - 4{\mathcal R_0}+4)}$ to become negative.
Since $\sqrt{\eps(\eps{\mathcal R_0}^2 - 4{\mathcal R_0}+4)}$ is only negative when $\eps < \eps^*$ and $\sqrt{\eps(\eps{\mathcal R_0}^2 - 4{\mathcal R_0}+4)} < 0$ implies non-negative imaginary components of the eigenvalue of $J(\hat{S},\hat{I})$, which in turn implies dampened oscillations when the model is approaching the EE.
Thus $\eps < \eps^*$ is a necessary condition for dampened oscillations in the model while approaching the EE.

\item

% !Rnw root = main.Rnw


%%EXPRESSIONS OF PERIOD and E-FOLDING TIME%%
Assuming $\epsilon < \epsilon^*$, the period of damped oscillations to the EE and the $e$-folding time of decay of the amplitude of oscilations is determined by analyzing the complex eigenvalue of the Jacobian of the system evaluated at the EE.\\

The Jacobian of the system is:
$$ J(S, I) =
\begin{bmatrix}
    -\beta I - \mu  & -\beta S                 & 0\\
    \beta I         & \beta S - (\mu + \gamma)  & 0\\
    0               & \gamma                    & -\mu\\
\end{bmatrix} 
$$

To simplify, the above Jacobian can be reduced to the upper left $ 2 \times 2$ matrix (since the system can be reduced to the first two equations) and evaluated at the EE $(\hat{S}, \hat{I})$. The characteristic polynomial is computed by substracting $\lambda$ on the diagonal and calculating the determinant.\\
So simplifying $J(\hat{S}, \hat{I})$, the characteristic polynomial is:
\begin{equation}
  (-\beta \hat{I} - \mu - \lambda)(\beta \hat{S} - (\mu + \gamma) - \lambda) + \beta^2 \hat{S}\hat{I} = 0
\end{equation}
 The EE is $(\frac{\mu + \gamma}{\beta}, \frac{\mu}{\beta}(\mathcal R_0 - 1))$ so the characteristic polynomial at the EE is:
 
 \begin{align*}
  0 &= (-\beta \hat{I} - \mu - \lambda)(\beta \hat{S} - (\mu + \gamma) - \lambda) + \beta^2 \hat{S}\hat{I}\\
  0 &= (-\mu(\mathcal R_0 - 1) - \mu - \lambda)(- \lambda) + (\mu + \gamma) \mu(\mathcal R_0 - 1)\\
  0 &= \lambda^2 + (\mu \mathcal R_0)\lambda + (\mu + \gamma)\mu(\mathcal R_0 - 1)\\
\end{align*}

Solving for the eigenvalue, $\lambda$, using the quadratic equation:
\begin{equation}
  \lambda = -\frac{\mu \mathcal R_0}{2} \pm \frac {\sqrt{(\mu \mathcal R_0)^2-4(\mu + \gamma)\mu(\mathcal R_0 - 1)}}{2}
\end{equation}

Assuming that$(\mu \mathcal R_0)^2$ is relatively small, then this eigenvalue will be complex, $\lambda = -\frac{\mu \mathcal R_0}{2} \pm iM$. The period of damped oscillations, $T$, is $T = 2\pi\frac{1}{M}$ and the $e$-folding time, $t_e = \frac{1}{re(\lambda)}$, is $t_e = \frac{2}{\mu \mathcal R_0}$. \\

Furthermore if we assume that $(\mu \mathcal R_0)^2 \approx 0$, the above can be simplified to:

\begin{equation}
  \lambda \approx -\frac{\mu \mathcal R_0}{2} \pm i \sqrt{(\mu + \gamma)\mu(\mathcal R_0 - 1)}
\end{equation}

So the period of damped oscillations can be approximated as $T \approx \frac{2\pi}{\sqrt{(\mu + \gamma)\mu(\mathcal R_0 - 1)}}$. \par

The period of damped oscillations can observed in the following graph which uses $I(0) = 0.00025, S(0) = 0.1, \beta = 550, \gamma=365/7, \mu=1/70$. 

%%GRAPH EXAMPLE%%
\begin{knitrout}
\definecolor{shadecolor}{rgb}{0.969, 0.969, 0.969}\color{fgcolor}\begin{kframe}
\begin{alltt}
\hlstd{SIR.vector.field} \hlkwb{<-} \hlkwa{function}\hlstd{(}\hlkwc{t}\hlstd{,} \hlkwc{vars}\hlstd{,} \hlkwc{parms}\hlstd{=}\hlkwd{c}\hlstd{(}\hlkwc{beta}\hlstd{=}\hlnum{3}\hlstd{,}\hlkwc{gamma}\hlstd{=}\hlnum{1}\hlstd{,} \hlkwc{mu}\hlstd{=}\hlnum{0.05}\hlstd{)) \{}
\hlkwd{with}\hlstd{(}\hlkwd{as.list}\hlstd{(}\hlkwd{c}\hlstd{(parms, vars)), \{}
\hlstd{dx} \hlkwb{<-} \hlopt{-}\hlstd{beta}\hlopt{*}\hlstd{x}\hlopt{*}\hlstd{y} \hlopt{+} \hlstd{mu} \hlopt{-} \hlstd{mu}\hlopt{*}\hlstd{x} \hlcom{# dS/dt of SIR model}
\hlstd{dy} \hlkwb{<-} \hlstd{beta}\hlopt{*}\hlstd{x}\hlopt{*}\hlstd{y} \hlopt{-} \hlstd{gamma}\hlopt{*}\hlstd{y} \hlopt{-} \hlstd{mu}\hlopt{*}\hlstd{y} \hlcom{# dI/dt of SIR model}
\hlstd{dz} \hlkwb{<-} \hlstd{gamma}\hlopt{*}\hlstd{y} \hlopt{-} \hlstd{mu}\hlopt{*}\hlstd{z} \hlcom{#dR/dt of SIR model}
\hlstd{vec.fld} \hlkwb{<-} \hlkwd{c}\hlstd{(}\hlkwc{dx}\hlstd{=dx,} \hlkwc{dy}\hlstd{=dy,} \hlkwc{dz}\hlstd{=dz)}
\hlkwd{return}\hlstd{(}\hlkwd{list}\hlstd{(vec.fld))} \hlcom{# ode() requires a list}
\hlstd{\})}
\hlstd{\}}
\hlcom{##Plots the solution I(t) of the SIR model}
\hlstd{plot.It} \hlkwb{<-} \hlkwa{function}\hlstd{(}\hlkwc{ic}\hlstd{=}\hlkwd{c}\hlstd{(}\hlkwc{x}\hlstd{=}\hlnum{1}\hlstd{,}\hlkwc{y}\hlstd{=}\hlnum{0}\hlstd{,}\hlkwc{z}\hlstd{=}\hlnum{0}\hlstd{),} \hlkwc{tmax}\hlstd{=}\hlnum{1}\hlstd{,}
\hlkwc{times}\hlstd{=}\hlkwd{seq}\hlstd{(}\hlnum{0}\hlstd{,tmax,}\hlkwc{by}\hlstd{=tmax}\hlopt{/}\hlnum{500}\hlstd{),}
\hlkwc{func}\hlstd{,} \hlkwc{parms}\hlstd{,} \hlkwc{...} \hlstd{) \{}
\hlstd{It} \hlkwb{<-} \hlkwd{ode}\hlstd{(ic, times, func, parms)}
\hlkwd{lines}\hlstd{(times, It[,}\hlstr{"y"}\hlstd{],} \hlkwc{col}\hlstd{=}\hlstr{"blue"}\hlstd{,} \hlkwc{lwd}\hlstd{=}\hlnum{2}\hlstd{, ... )}
\hlstd{\}}

\hlstd{tmax} \hlkwb{<-} \hlnum{70} \hlcom{# end time for numerical integration of the ODE}
\hlcom{## draws the empty plot:}
\hlkwd{plot}\hlstd{(}\hlnum{0}\hlstd{,}\hlnum{0}\hlstd{,}\hlkwc{xlim}\hlstd{=}\hlkwd{c}\hlstd{(}\hlnum{0}\hlstd{,}\hlnum{70}\hlstd{),}\hlkwc{ylim}\hlstd{=}\hlkwd{c}\hlstd{(}\hlnum{0.00005}\hlstd{,}\hlnum{0.0006}\hlstd{),} \hlkwc{xaxs}\hlstd{=}\hlstr{"i"}\hlstd{,}
\hlkwc{type}\hlstd{=}\hlstr{"n"}\hlstd{,}\hlkwc{bty}\hlstd{=}\hlstr{"L"}\hlstd{,}\hlkwc{xlab}\hlstd{=}\hlstr{"Time (t)"}\hlstd{,}\hlkwc{ylab}\hlstd{=}\hlstr{"Prevalence (I)"}\hlstd{,}\hlkwc{las}\hlstd{=}\hlnum{1}\hlstd{)}

\hlcom{## Given initial conditions and parameter values:}
\hlstd{I0} \hlkwb{<-} \hlnum{0.00025}
\hlstd{S0} \hlkwb{<-} \hlnum{0.1}

\hlstd{\{}
\hlkwd{plot.It}\hlstd{(}\hlkwc{ic}\hlstd{=}\hlkwd{c}\hlstd{(}\hlkwc{x}\hlstd{=S0,}\hlkwc{y}\hlstd{=I0,}\hlkwc{z}\hlstd{=}\hlnum{0}\hlstd{),} \hlkwc{tmax}\hlstd{=tmax,}
\hlkwc{func}\hlstd{=SIR.vector.field,}
\hlkwc{parms}\hlstd{=}\hlkwd{c}\hlstd{(}\hlkwc{beta}\hlstd{=}\hlnum{550}\hlstd{,}
        \hlkwc{gamma}\hlstd{=}\hlnum{365}\hlopt{/}\hlnum{7}\hlstd{,} \hlcom{#mean infectious period is 7 days in a year}
        \hlkwc{mu}\hlstd{=}\hlnum{1}\hlopt{/}\hlnum{70} \hlcom{#the average lifespan in the population is 70 years}
        \hlstd{)}
\hlstd{)}
\hlstd{\}}
\end{alltt}
\end{kframe}
\includegraphics[width=\maxwidth]{figure/SIRmodel-1} 

\end{knitrout}

\begin{knitrout}
\definecolor{shadecolor}{rgb}{0.969, 0.969, 0.969}\color{fgcolor}\begin{kframe}
\begin{alltt}
\hlstd{beta} \hlkwb{<-} \hlnum{550}
\hlstd{gamma} \hlkwb{<-} \hlnum{365}\hlopt{/}\hlnum{7}
\hlstd{mu} \hlkwb{<-} \hlnum{1}\hlopt{/}\hlnum{70}

\hlstd{R0} \hlkwb{<-} \hlstd{beta}\hlopt{/}\hlstd{(gamma}\hlopt{+}\hlstd{mu)}
\hlstd{A} \hlkwb{<-} \hlstd{(gamma}\hlopt{+}\hlstd{mu)}\hlopt{*}\hlstd{mu}\hlopt{*}\hlstd{(R0}\hlopt{-}\hlnum{1}\hlstd{)}
\hlstd{epsilon} \hlkwb{<-} \hlstd{mu}\hlopt{/}\hlstd{(gamma}\hlopt{+}\hlstd{mu)}
\hlstd{period} \hlkwb{<-} \hlnum{2}\hlopt{*}\hlstd{pi}\hlopt{/}\hlkwd{Im}\hlstd{(}\hlkwd{sqrt}\hlstd{(}\hlkwd{as.complex}\hlstd{((mu}\hlopt{*}\hlstd{R0)}\hlopt{^}\hlnum{2}\hlopt{-}\hlnum{4}\hlopt{*}\hlstd{A))}\hlopt{/}\hlnum{2}\hlstd{)} \hlcom{#equation from previous analysis}
\hlstd{efold} \hlkwb{<-} \hlnum{2}\hlopt{/}\hlstd{(mu}\hlopt{*}\hlstd{R0)}
\hlstd{R0}
\end{alltt}
\begin{verbatim}
## [1] 10.54506
\end{verbatim}
\begin{alltt}
\hlstd{epsilon}
\end{alltt}
\begin{verbatim}
## [1] 0.0002738976
\end{verbatim}
\begin{alltt}
\hlstd{period}
\end{alltt}
\begin{verbatim}
## [1] 2.356981
\end{verbatim}
\begin{alltt}
\hlstd{efold}
\end{alltt}
\begin{verbatim}
## [1] 13.27636
\end{verbatim}
\end{kframe}
\end{knitrout}

In the graph, the distance from peak to peak is slightly larger than 2 years, indicating approximately a 2-year cycle of oscillations. This is confirmed with the above calculation where the period is 2.356 years.
\item

As $\mathcal R_0$ is increased from 0 to $\infty$, using the dimensionless forms of the differential equations, it is apparent that one of the bifurcations of $\mathcal R_0$ is $\mathcal R_0 = 1$ (which is the transcritical bifurcation).

Consider $\frac{dI}{d\tau} = I(\mathcal R_0 S - 1)$ when and initially $S\approx 1$ and $I \ll 1$. Then an epidemic can occur if $\frac{dI}{d\tau} > 0$ (meaning that the number of infectious people increases).
Hence, for an epidemic (i.e. convergence to the EE):
\begin{align*}
  I(\mathcal R_0 S - 1) &> 0\\
  \mathcal R_0 - 1 &> 0\\
  \mathcal R_0 &> 1
\end{align*}

The two ``bifurcations" that yield biologically relevant changes can be determined by looking at the boundaries of $\mathcal R_0$ in terms of $\epsilon^*$ threshold of damped oscilation in part (h):

\begin{align*}
  \epsilon &= \frac{4(\mathcal R_0 - 1)}{(\mathcal R_0)^2}\\
  \epsilon (\mathcal R_0)^2 - 4 \mathcal R_0 + 4 &= 0\\
\end{align*}

Using the quadratic formula, it follows that there are two possible $\mathcal R_0$ values:
\begin{align*}
  \mathcal R_0 &= \frac{4 \pm \sqrt{16 - 16 \epsilon}}{2 \epsilon} \\
\end{align*}

To visualize these results, a value of $\epsilon = \frac{8}{9}$ will be used since it is a biologically reasonable value (and the corresponding ``bifurcations" are also biologically valid). Using this value we get $\mathcal R_0 = 1.5 , 3$.\par
When $0 < \mathcal R_0 < 1$, then solutions will converge to the DFE. When $1 < \mathcal R_0 < 1.5$, then dynamics will be less oscillatory, meaning that recurrent epidemics will dampen out relatively quickly. This is also similar to the dynamics of the system when $3 < \mathcal R_0 < Z$, where $Z$ is large. 
When $1.5 < \mathcal R_0 < 3$, the system will likely show slower damped oscillatory dynamics, where more reccurent epidemics are possible before the system converges to the EE. 

\begin{knitrout}
\definecolor{shadecolor}{rgb}{0.969, 0.969, 0.969}\color{fgcolor}\begin{kframe}
\begin{alltt}
\hlcom{##Dimensionless form (only parameters are epsilon and \textbackslash{}mathcal R_0)##}
\hlstd{SIR.vector.field} \hlkwb{<-} \hlkwa{function}\hlstd{(}\hlkwc{t}\hlstd{,} \hlkwc{vars}\hlstd{,} \hlkwc{parms}\hlstd{=}\hlkwd{c}\hlstd{(}\hlkwc{epsilon} \hlstd{=} \hlnum{8}\hlopt{/}\hlnum{9}\hlstd{,} \hlkwc{R0} \hlstd{=} \hlnum{5}\hlstd{)) \{}
\hlkwd{with}\hlstd{(}\hlkwd{as.list}\hlstd{(}\hlkwd{c}\hlstd{(parms, vars)), \{}
\hlstd{dx} \hlkwb{<-} \hlstd{epsilon}\hlopt{*}\hlstd{(}\hlnum{1}\hlopt{-}\hlstd{x)}\hlopt{-}\hlstd{R0}\hlopt{*}\hlstd{x}\hlopt{*}\hlstd{y} \hlcom{# dS/dt of SIR model}
\hlstd{dy} \hlkwb{<-} \hlstd{R0}\hlopt{*}\hlstd{x}\hlopt{*}\hlstd{y} \hlopt{-} \hlstd{y} \hlcom{# dI/dt of SIR model}
\hlstd{dz} \hlkwb{<-} \hlstd{(}\hlnum{1}\hlopt{-}\hlstd{epsilon)}\hlopt{*}\hlstd{y} \hlopt{-} \hlstd{epsilon}\hlopt{*}\hlstd{z} \hlcom{#dR/dt of SIR model}
\hlstd{vec.fld} \hlkwb{<-} \hlkwd{c}\hlstd{(}\hlkwc{dx}\hlstd{=dx,} \hlkwc{dy}\hlstd{=dy,} \hlkwc{dz}\hlstd{=dz)}
\hlkwd{return}\hlstd{(}\hlkwd{list}\hlstd{(vec.fld))}
\hlstd{\})}
\hlstd{\}}

\hlcom{##S(I) plotting function##}
\hlstd{plot.SI} \hlkwb{<-} \hlkwa{function}\hlstd{(}\hlkwc{ic}\hlstd{=}\hlkwd{c}\hlstd{(}\hlkwc{x}\hlstd{=}\hlnum{1}\hlstd{,}\hlkwc{y}\hlstd{=}\hlnum{0}\hlstd{,}\hlkwc{z}\hlstd{=}\hlnum{0}\hlstd{),} \hlkwc{tmax}\hlstd{=}\hlnum{1}\hlstd{,}
\hlkwc{times}\hlstd{=}\hlkwd{seq}\hlstd{(}\hlnum{0}\hlstd{,tmax,}\hlkwc{by}\hlstd{=tmax}\hlopt{/}\hlnum{500}\hlstd{),}
\hlkwc{func}\hlstd{,} \hlkwc{parms}\hlstd{,} \hlkwc{...} \hlstd{) \{}
\hlstd{St} \hlkwb{<-} \hlkwd{ode}\hlstd{(ic, times, func, parms)}
\hlkwd{lines}\hlstd{(St[,}\hlstr{"x"}\hlstd{], St[,}\hlstr{"y"}\hlstd{],} \hlkwc{col}\hlstd{=}\hlstr{"blue"}\hlstd{,} \hlkwc{lwd}\hlstd{=}\hlnum{1.5}\hlstd{, ... )}
\hlstd{\}}

\hlcom{##Various R0 values##}
\hlstd{R0vals} \hlkwb{<-} \hlkwd{c}\hlstd{(}\hlnum{0.9}\hlstd{,} \hlnum{1.1}\hlstd{,} \hlnum{2.25}\hlstd{,} \hlnum{5}\hlstd{)}
\hlcom{#intial conditions}
\hlstd{S0} \hlkwb{<-} \hlkwd{seq}\hlstd{(}\hlnum{0}\hlstd{,}\hlnum{0.9}\hlstd{,}\hlkwc{by}\hlstd{=}\hlnum{0.1}\hlstd{)[}\hlopt{-}\hlnum{1}\hlstd{]}
\hlstd{I0} \hlkwb{<-} \hlkwd{seq}\hlstd{(}\hlnum{0.9}\hlstd{,}\hlnum{0}\hlstd{,} \hlkwc{by}\hlstd{=}\hlopt{-}\hlnum{0.1}\hlstd{)[}\hlopt{-}\hlnum{10}\hlstd{]}
\hlkwd{par}\hlstd{(}\hlkwc{mfrow} \hlstd{=} \hlkwd{c}\hlstd{(}\hlnum{2}\hlstd{,}\hlnum{2}\hlstd{))} \hlcom{#Setting up the subplots}
\hlkwa{for} \hlstd{(i} \hlkwa{in} \hlnum{1}\hlopt{:}\hlkwd{length}\hlstd{(R0vals)) \{}
  \hlstd{pars} \hlkwb{<-}  \hlkwd{c}\hlstd{(}\hlkwc{epsilon} \hlstd{=} \hlnum{1}\hlopt{/}\hlnum{9}\hlstd{,} \hlkwc{R0} \hlstd{= R0vals[i])}
  \hlstd{title} \hlkwb{<-} \hlkwd{paste}\hlstd{(}\hlstr{"R_0 ="}\hlstd{, R0vals[i])} \hlcom{#labelling R0 for each plot}
  \hlkwd{plot}\hlstd{(}\hlkwc{x}\hlstd{=}\hlnum{0}\hlstd{,} \hlkwc{y}\hlstd{=}\hlnum{0}\hlstd{,} \hlkwc{type} \hlstd{=} \hlstr{"n"}\hlstd{,} \hlkwc{xlim} \hlstd{=} \hlkwd{c}\hlstd{(}\hlnum{0}\hlstd{,}\hlnum{1}\hlstd{),} \hlkwc{ylim} \hlstd{=} \hlkwd{c}\hlstd{(}\hlnum{0}\hlstd{,}\hlnum{1}\hlstd{),} \hlcom{#empty subplot}
       \hlkwc{xlab} \hlstd{=} \hlstr{"Susceptible (S)"}\hlstd{,} \hlkwc{ylab}\hlstd{=} \hlstr{"Infected (I)"}\hlstd{,} \hlkwc{main}\hlstd{=title)}
  \hlkwd{lines}\hlstd{(}\hlkwc{x}\hlstd{=}\hlkwd{c}\hlstd{(}\hlnum{1}\hlstd{,}\hlnum{0}\hlstd{),}\hlkwc{y}\hlstd{=}\hlkwd{c}\hlstd{(}\hlnum{0}\hlstd{,}\hlnum{1}\hlstd{),}\hlkwc{col}\hlstd{=}\hlstr{"grey"}\hlstd{,}\hlkwc{lty}\hlstd{=}\hlstr{"dotted"}\hlstd{)}
  \hlkwa{for} \hlstd{(j} \hlkwa{in} \hlnum{1}\hlopt{:}\hlkwd{length}\hlstd{(S0)) \{}
    \hlkwd{plot.SI}\hlstd{(}\hlkwc{ic}\hlstd{=}\hlkwd{c}\hlstd{(}\hlkwc{x}\hlstd{=S0[j],}\hlkwc{y}\hlstd{=I0[j],}\hlkwc{z}\hlstd{=}\hlnum{0}\hlstd{),} \hlkwc{tmax}\hlstd{=tmax,}
            \hlkwc{func}\hlstd{=SIR.vector.field,}
            \hlkwc{parms}\hlstd{=pars}
  \hlstd{)}
  \hlstd{\}}
\hlstd{\}}
\end{alltt}
\end{kframe}
\includegraphics[width=\maxwidth]{figure/phase_portraits-1} 

\end{knitrout}
\item There are no diseases that display recurrent epidemics for which the SIR model with vital dynamics is adequate to explain the observed epidemic dynamics. From results in parts (g) and (h), given that $\R_0 > 1$ (such that an epidemic occurs) the EE is GAS and that for all initial conditions, $I(0) > 0, S(0) > 0$, the system approaches the EE with damped oscillations. Additionally, by observing the Jacobian evaluated at the EE, the complex eigenvalues have negative real part and non-zero imaginary part, implying that the dynamics are always oscilatory. Thus recurrent epidemics with no evidence of damping out (like measles) cannot be explained by the SIR model with vital dynamics. 

\end{enumerate}
\printbibliography

%%\newpage
%%\bibliographystyle{vancouver}
%%\bibliography{DavidEarn,MyPubs}

\bigskip\vfill

\centerline{\bf--- END OF ASSIGNMENT ---}

\bigskip
Compile time for this document:
\today\ @ \thistime

\end{document}
