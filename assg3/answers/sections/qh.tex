%explanation
Since the EE is GAS, any oscillations will be damped and approaching the EE.
Oscillations will occur if the imaginary parts of the eigenvalues of the Jacobian, evaluated at the EE $(\hat{S},\hat{I}) = (\frac{1}{\mathcal R_0},\eps(1-\frac{1}{\mathcal R_0})$, are non-zero.
Since the Jacobian is a $2\times 2$ matrix, the eigenvalues are the roots of the equation $\lambda^2 - T\lambda + D = 0$ where T and D are the trace and determinant of the Jacobian respectively.
The roots can be found using the quadratic formula $\frac{-b \pm \sqrt{b^2-4ac}}{2a}$.
The imaginary part of the roots will only be non-zero if $\sqrt{b^2-4ac} < 0$, thus proving $\sqrt{b^2-4ac} < 0$ implies damped oscillations approaching the EE.\\
The Jacobian at the EE is
\begin{align}
J &=
\begin{bmatrix}
    \frac{\partial S}{\partial S} & \frac{\partial S}{\partial I}\\
    \frac{\partial I}{\partial S} & \frac{\partial I}{\partial I}
\end{bmatrix}\\
&=
\begin{bmatrix}
    -\beta I - \mu & -\beta S\\
    \beta I &\beta S - (\gamma+\mu)
\end{bmatrix}\\
J(\hat{S},\hat{I}) &= J(\frac{1}{\mathcal R_0},\eps(1-\frac{1}{\mathcal R_0})\\
&=
\begin{bmatrix}
    -\beta \eps(1-\frac{1}{\mathcal R_0}) - \mu & -\beta \frac{1}{\mathcal R_0}\\
    \beta \eps(1-\frac{1}{\mathcal R_0}) &\beta \frac{1}{\mathcal R_0} - (\gamma+\mu)
\end{bmatrix}\\
& \textrm{Next substitute $\beta, \gamma, \mu$ using equations \ref{gamma},\ref{beta},\ref{mu}}\nonumber\\
&=
\begin{bmatrix}
    -(\gamma+\mu){\mathcal R_0} \eps(1-\frac{1}{\mathcal R_0}) - \eps(\gamma+\mu) & -(\gamma+\mu){\mathcal R_0} \frac{1}{\mathcal R_0}\\
    (\gamma+\mu){\mathcal R_0} \eps(1-\frac{1}{\mathcal R_0}) &(\gamma+\mu){\mathcal R_0} \frac{1}{\mathcal R_0} - (\gamma+\mu)
\end{bmatrix}\\
&= (\gamma+\mu)
\begin{bmatrix}
    -\eps {\mathcal R_0}  & -1\\
    \eps({\mathcal R_0}-1) & 0
\end{bmatrix}
\end{align}
Next the trace $T$, determinant $D$, and the terms of the quadratic formula $a,b,c$ are
\begin{align}
    T &= -\eps {\mathcal R_0} + 0\\
    D &= (-\eps {\mathcal R_0})(0) - (-1)(\eps({\mathcal R_0}-1)\\
      &= \eps({\mathcal R_0}-1)\\
    0 &= \lambda^2 - T \lambda + D \implies a = 1\ \ b = -T\ \ c = D\\
    b &= \eps {\mathcal R_0}\\
    c &= \eps({\mathcal R_0}-1)\\
    \sqrt{b^2-4ac} &= \sqrt{\eps^2 {\mathcal R_0}^2 - 4(1)(\eps({\mathcal R_0}-1))}\\
                   &= \sqrt{\eps^2 {\mathcal R_0}^2 - 4(\eps({\mathcal R_0}+4\eps}\\
                   &= \sqrt{\eps(\eps{\mathcal R_0}^2 - 4{\mathcal R_0}+4)}\label{bsqr}
\end{align}
Now if we show that the when the inequality
\begin{equation}\label{eqee}
    \sqrt{\eps(\eps{\mathcal R_0}^2 - 4{\mathcal R_0}+4)} < 0
\end{equation}
holds it implies $\eps < \eps^* = \frac{4{\mathcal R_0}-1}{{\mathcal R_0}^2}$, then this proves that the approach to EE via damped oscillations occurs iff $\eps < \eps^*$
\begin{align}
    \sqrt{\eps(\eps{\mathcal R_0}^2 - 4{\mathcal R_0}+4)} &< 0\\
    \sqrt{\frac{4{\mathcal R_0}-1}{{\mathcal R_0}^2}(\frac{4({\mathcal R_0}-1)}{{\mathcal R_0}^2}{\mathcal R_0}^2 - 4{\mathcal R_0}+4)} &< 0\\
    \sqrt{\frac{4{\mathcal R_0}-1}{{\mathcal R_0}^2}(4({\mathcal R_0}-1) - 4{\mathcal R_0}+4)} &< 0\\
    \sqrt{\frac{4{\mathcal R_0}-1}{{\mathcal R_0}^2}(4{\mathcal R_0}-4 - 4{\mathcal R_0}+4)} &< 0\\
    \sqrt{\frac{4{\mathcal R_0}-1}{{\mathcal R_0}^2}(0)} &< 0\\
    0 &< 0 \textrm{which is false}
\end{align}
Since \ref{eeqe} is 0 when $\eps = \eps^*$, increasing $\eps$ will cause $\sqrt{\eps(\eps{\mathcal R_0}^2 - 4{\mathcal R_0}+4)}$ to become positive and decreasing $\eps$ will cause $\sqrt{\eps(\eps{\mathcal R_0}^2 - 4{\mathcal R_0}+4)}$ to become negative.
Since $\sqrt{\eps(\eps{\mathcal R_0}^2 - 4{\mathcal R_0}+4)}$ is only negative when $\eps < \eps^*$ and $\sqrt{\eps(\eps{\mathcal R_0}^2 - 4{\mathcal R_0}+4)} < 0$ implies non-negative imaginary components of the eigenvalue of $J(\hat{S},\hat{I})$, which in turn implies dampened oscillations when the model is approaching the EE.
Thus $\eps < \eps^*$ is a necessary condition for dampened oscillations in the model while approaching the EE.
