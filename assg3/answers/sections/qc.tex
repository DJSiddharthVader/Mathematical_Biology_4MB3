%d/dtau
First we express $dt$ in terms of $\tau$
\begin{align}
    \tau &= t(\gamma + \mu)\nonumber\\
    d\tau &= dt(\gamma + \mu)\nonumber\\
    \frac{d\tau}{dt} &= (\gamma + \mu)\nonumber\\
    \frac{d\tau}{dt}\frac{d}{d\tau} &= \frac{d}{dt}\nonumber\\
    (\gamma+\mu)\frac{d}{d\tau} &= \frac{d}{dt}\nonumber\\
    \frac{d}{d\tau} &= \frac{1}{(\gamma+\mu)}\frac{d}{dt}\label{tau}
\end{align}
%isloate gamma,beta,mu
Next we isolate $\gamma,\beta,\mu$,
\begin{align}
    \eps &= \frac{\mu}{\gamma+\mu}\nonumber\\
    \eps(\gamma + \mu) &= \mu &\gamma+\mu = \frac{\mu}{\eps} \nonumber\\
    \eps \gamma + \eps \mu &= \mu\nonumber\\
    \eps \gamma &= \mu(1-\eps)\nonumber\\
    \gamma &= \frac{\mu}{\eps}(1-\eps)\nonumber\\
    \gamma &= (\gamma+\mu)(1-\eps)\label{gamma}\\
    \textrm{From (2b) it follows that}\ \beta &= (\gamma+\mu){\mathcal R_0}\label{beta}\\
    \textrm{From (2c) it follows that}\ \mu &= (\gamma+\mu)\eps\label{mu}
\end{align}
%dS/dt
Next, expressing $\frac{dS}{dt}$ in terms of $\tau$ using \ref{tau} and substituting equations \ref{gamma},\ref{beta} and \ref{mu} gives
\begin{align}
    \frac{dS}{d\tau} &= \frac{1}{\gamma+\mu}\frac{dS}{dt}\nonumber\\
     &= \frac{1}{\gamma+\mu}[\mu-\beta SI -\mu S]\nonumber\\
     &= \frac{1}{\gamma+\mu}[\mu(1-S)-\beta SI]\\
     &= \frac{1}{\gamma+\mu}[\eps(\gamma+\mu)(1-S)- (\gamma+\mu){\mathcal R_0}SI]\nonumber\\
     &= \frac{1}{\gamma+\mu}(\gamma+\mu)[\eps(1-S)- {\mathcal R_0}SI]\nonumber\\
     &= \eps(1-S)- {\mathcal R_0}SI
\end{align}
%dI/dt
Solving for $\frac{dI}{d\tau}$ using \ref{tau}, \ref{gamma}, \ref{beta} and \ref{mu} gives
\begin{align}
    \frac{dI}{d\tau} &= \frac{1}{\gamma+\mu}\frac{dI}{dt}\nonumber\\
     &= \frac{1}{\gamma+\mu}[\beta SI -\gamma I-\mu I]\nonumber\\
     &= \frac{1}{\gamma+\mu}[(\gamma+\mu){\mathcal R_0}SI -(\gamma+\mu)(1-\eps) I-(\gamma+\mu)\eps I]\nonumber\\
     &= \frac{1}{\gamma+\mu}(\gamma+\mu)[{\mathcal R_0}SI -(1-\eps) I-\eps I]\nonumber\\
     &= {\mathcal R_0}SI -(1-\eps) I-\eps I\nonumber\\
     &= {\mathcal R_0}SI -(1-2\eps) I
\end{align}
%dR/dt
Finally, solving for $\frac{dR}{d\tau}$ using \ref{tau}, \ref{gamma}, \ref{beta} and \ref{mu} gives
\begin{align*}
    \frac{dR}{d\tau} &= \frac{1}{\gamma+\mu}\frac{dR}{dt}\nonumber\\
     &= \frac{1}{\gamma+\mu}[\gamma I - \mu R]\nonumber\\
     &= \frac{1}{\gamma+\mu}[(\gamma+\mu)(1-\eps)I - (\gamma+\mu)\eps R]\nonumber\\
     &= \frac{1}{\gamma+\mu}(\gamma+\mu)[(1-\eps)I - \eps R]\nonumber\\
     &= (1-\eps)I - \eps R
\end{align*}
%Biological meanings
%1/(mu+gamma) is average time of infection
The biological meanings of $\tau, {\mathcal R_0}$ and $\eps$ are
\begin{itemize}
    \item $\tau$ is the average proportion of the population infected
    \item ${\mathcal R_0}$ is the number of secondary infections per infection
    \item $\eps$ is the mortality rate for the infected.
    %\item $\eps$ is the average period being infected until death or recovery
    %\item $\eps$ is the ratio of new people who can be infected (births) to people who can no longer be infected (recover or die), or the rate at which the succeptible class grows.
\end{itemize}
%justification for tau,epsilon,rnot

%disease epsilon
In the UK the morality rate from penumonia between 2001-2010 was estimated to be $0.0214\%$ of people\cite{pneu}.
Ebola Virus Diesase is estimated by the WHO to have an average case fatality rate of $50\%$\cite{ebola}.
The CDC estimates measles to have a mortality rate between $0.1\%$ to $0.2\%$ \cite{meas}.
%For ebola the mean infectious period ($\eps$) has been estimated to be $5.33\pm 4.03$ days from wet symptom onset until death.\cite{ebola}.
%For influenza the mean infectious period ($\eps$) has been estimated to be $1.0$ days until death.\cite{flu}. For other diseases $\eps$ appears to be on the order of days to weeks.
