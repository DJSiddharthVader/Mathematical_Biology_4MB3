Local asymptotic stability can be derived logically from the definition of $R_0$. When $R_0 < 1$, this means that the number of secondary infections per infection is less than 1. Thus, if the disease has just started, the number of infected will immediately go back to 0 or the disease free equilibrium.\\
On the other hand if $R_0 > 1$, the disease will spread once it starts. The number of infected (I) can never reach 1 either since $\epsilon$ (mortality rate) exists, so there is also an upper limit. Ergo, the equilibrium in between (0,1), or the endemic equilibrium, must be asymptotically stable when $R_0 > 1$.\\
A more mathematically detailed proof is done in parts f and g, proving global asymptotic stability under these conditions.